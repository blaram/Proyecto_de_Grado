\documentclass[11pt,openany,oneside,letterpaper]{book_tesis}
\usepackage[spanish]{babel}
%\usepackage[utf8]{inputenc}
\usepackage{fontspec}			% Necesario para XeLaTeX o LuaLaTeX
\usepackage{setspace}           % Interlineado
\usepackage{titletoc}			% Formato del indice
\usepackage{titlesec}			% Formato del indice
\usepackage[left=2.54cm, right=2.54cm, top=2.54cm, bottom=2.54cm]{geometry}

\usepackage{graphicx}  % Para insertar imágenes
\usepackage{caption}   % Para personalizar leyendas
\usepackage{float}     % Para control de posicionamiento de figuras
\usepackage{booktabs}   % Para líneas horizontales en
\usepackage{array}      % Para personalizar alineación en columnas
\usepackage{longtable} % Paquete para tablas largas
\usepackage{multirow} % para unión de filas en las tablas

\usepackage{changepage} % Para ajustar márgenes

\usepackage{listings} % Para mostrar código fuente
\usepackage{xcolor}  % Para colores

\usepackage{amssymb}
\usepackage{amsfonts}
\usepackage{amsmath}
\usepackage[backend=biber, style=apa]{biblatex} % Para usar bibliografia
\usepackage{csquotes}
\usepackage{enumitem}			% para las listas 
\usepackage[nopatch=item]{microtype}		% Para que no salga error de polyglossia
\usepackage{lipsum} % Carga el paquete lipsum
\usepackage{microtype}
\usepackage{hyperref} % Para enlaces a los capitulos secciones en el indice
\usepackage{changepage} % Para ajustar márgenes

% \usepackage{showframe}                % Para ver lineas de margenes

\definecolor{gris}{gray}{0.95}

\addbibresource{parte_final/bibliografia.bib}

\setmainfont{georgia} 			% Configura la fuente principal como Georgia

\setlength{\parskip}{0.1mm}           % Espacio entre párrafos
\setlength{\parindent}{36.1315pt}     % Indentación de los párrafos
\setlength{\baselineskip}{0em} % Ajusta el espacio entre líneas

\pagestyle{myheadings}

% Cambiar "Cuadro" a "Tabla"
\addto\captionsspanish{\renewcommand{\tablename}{Tabla}}
% Cambiar "Índice de Cuadros" a "Índice de Tablas"
\addto\captionsspanish{\renewcommand{\listtablename}{ÍNDICE DE TABLAS}}
% Cambiar "Índice de figuras" a "ÍNDICE DE FIGURAS"
\addto\captionsspanish{\renewcommand{\listfigurename}{ÍNDICE DE FIGURAS}}

\assignpagestyle{\chapter}{empty} % Para que las hojas de los capítulos no sean numeradas

% Ajustar el espaciado vertical entre los capítulos en el índice
\titlecontents{chapter}[0pt]{\addvspace{1pt}\bfseries}{\thecontentslabel.\hspace{1.6em}}{}{\titlerule*[0.5pc]{.}\contentspage}
\titlecontents{section}[0pt]{}{\thecontentslabel.\hspace{1em}}{}{\titlerule*[0.5pc]{.}\contentspage}
\titlecontents{subsection}[28pt]{}{\thecontentslabel.\hspace{1em}}{}{\titlerule*[0.5pc]{.}\contentspage}

% Cambiar los títulos de los capítulos a mayúsculas
% \titleformat{\chapter}[display]
%{\normalfont\fontsize{13pt}{42pt}\selectfont\bfseries}{\MakeUppercase{\chaptertitlename}\ \thechapter}{-110pt}{\fontsize{13pt}{70pt}\selectfont}

\titleformat{\chapter}
{\normalfont\fontsize{13pt}{60pt}\selectfont\bfseries}{\thechapter}{24pt}{\fontsize{13pt}{0pt}\selectfont}

\titleformat{\section}{\bfseries\normalsize}{\thesection.}{16pt}{}
\titleformat{\subsection}{\bfseries\normalsize}{\thesubsection.}{9pt}{}

\titlespacing*{\chapter}{0pt}{*0}{0pt}
\titlespacing*{\section}{0pt}{*0}{0pt}
\titlespacing*{\subsection}{0pt}{*0}{0pt}

% Definir un nuevo estilo de numeración para los capítulos
% \renewcommand{\thechapter}{\Roman{chapter}}

% Configuración de hyperref para desactivar colores y cuadros
\hypersetup{
	colorlinks=false, % Desactiva el color de los enlaces
	pdfborder={0 0 0}, % Desactiva los cuadros alrededor de los enlaces
}

\captionsetup[figure]{
	margin={1.27cm,0cm},
	justification=RaggedRight, % Alineación a la izquierda
	singlelinecheck=false,     % Permite leyendas de múltiples líneas
	labelfont=bf,              % Fuente de etiqueta en negrita
	textfont=it,         	% Fuente en cursiva para texto
	labelsep=newline,           % Separador es un punto
}

\captionsetup[table]{
	margin={1.27cm,0cm},
	labelfont=bf,
	labelsep=newline,
	justification=RaggedRight,
	singlelinecheck=false,     % Permite leyendas de múltiples líneas
	textfont=it,      % Texto de la leyenda en cursiva
	% position=top,     % Leyenda arriba de la tabla
	% skip=0pt          % Sin espacio entre la leyenda y la tabla
}

\captionsetup[longtable]{
	margin={1.27cm,0cm},
	labelfont=bf,
	labelsep=newline,
	justification=RaggedRight,
	singlelinecheck=false,     % Permite leyendas de múltiples líneas
	textfont=it,      % Texto de la leyenda en cursiva
	% position=top,     % Leyenda arriba de la tabla
	skip=17pt          % Sin espacio entre la leyenda y la tabla
}

\makeatletter
\renewcommand\listoffigures{%
	\chapter*{\hfill\bfseries\normalsize ÍNDICE DE FIGURAS\hfill} % Título centrado
	\addcontentsline{toc}{chapter}{\listfigurename}%
	\noindent\hspace{0.6cm}\textbf{Fig.} \hfill \textbf{Descripción} \hfill \textbf{Pag.} \hfill % Encabezados
	\par\vspace{-0.2cm} % Espacio entre encabezados y el listado
	\@starttoc{lof}%
}
\makeatother

% Modificar el formato del índice de tablas
\makeatletter
\renewcommand\listoftables{%
	\chapter*{\hfill\bfseries\normalsize ÍNDICE DE TABLAS\hfill} % Título centrado
	\addcontentsline{toc}{chapter}{\listtablename}%
	\noindent\hspace{0.6cm}\textbf{Tabla} \hfill \textbf{Descripción} \hfill \textbf{Pag.} \hfill % Encabezados
	\par\vspace{-0.2cm} % Espacio entre encabezados y el listado
	\@starttoc{lot}%
}
\makeatother

\lstset{
	language=Python,
	basicstyle=\ttfamily\small, % Formato del código
	frame=single, % Marco alrededor del código
	tabsize=3,
	keywordstyle=\color{violet}\bfseries,  % Palabras clave en azul y negrita
	commentstyle=\color{olive},  % Comentarios en verde
	stringstyle=\color{cyan},  % Cadenas de texto en rojo
	backgroundcolor=\color{gris},  % Fondo gris claro
	breakatwhitespace=false,         % Activarlo para que los saltos automáticos solo se apliquen en los espacios en blanco
	breaklines=true,                 % Activa el salto de línea automático
	%xleftmargin=2.54cm, % Márgenes desde la izquierda
	%framexleftmargin=2.54cm, % Espacio para el marco
}


\begin{document}
	\spacing{2}                     % Interlineado
	\frontmatter
	\pagenumbering{gobble} % Eliminar numeración de página del índice
	\begin{titlepage}
	\begin{center}
		{\textbf{UNIVERSIDAD MAYOR DE SAN ANDRÉS}}\\
		{\textbf{FACULTAD DE CIENCIAS PURAS Y NATURALES}}\\
		{\textbf{CARRERA DE INFORMÁTICA}}\\
		\vspace{5mm}
		\begin{figure}[h]
			\centering
			\includegraphics[scale=1.09]{imagenes/Logo_UMSA.png}
		\end{figure}
		\vspace{3mm}
		{\textbf{PROYECTO DE GRADO}}\\
		
		{\textbf{SOFTWARE DE LOGÍSTICA Y GESTIÓN DE BUSES PARA TRANSPORTE DE PASAJEROS Y ENVÍO DE ENCOMIENDAS.\\
		CASO: EMPRESA DE TRANSPORTES CALI INTERNACIONAL}}\\
		{Proyecto de Grado para obtener el Título de Licenciatura en Informática}\\
		Mención: Ingeniería de Sistemas Informáticos\\
		
		\textbf{POSTULANTE:} BLADIMIR WILSON RAMOS ESCOBAR\\
		\textbf{TUTOR:} Ph. D. FRANZ CUEVAS QUIROZ\\
		\textbf{NUESTRA SEÑORA DE LA PAZ – BOLIVIA}\\
		\textbf{2025}\\
	\end{center}
\end{titlepage}
	\thispagestyle{empty}
	\sloppy
	% Dedicatoria
	%% Contenido del resumen
\thispagestyle{plain}

\vspace*{4cm} % Ajusta este espacio vertical según necesites

\begin{flushright}
	{\bfseries\fontsize{13pt}{16pt}\selectfont DEDICATORIA}
\end{flushright}

%\vspace{2cm} % Espacio entre el título y el texto

\begin{flushright}
	\textit{A mis padres, por ser el pilar fundamental de mi vida y por enseñarme con su ejemplo el valor del esfuerzo y la perseverancia. En especial, a mi querida madre Celia (+), que desde el cielo me acompaña e ilumina mi camino; por ser la única que confiaba en mí, su amor infinito sigue guiando cada uno de mis pasos.}
	
	\vspace{0.2cm}
	
	\textit{A mi amada esposa Rosmery, por su paciencia, apoyo incondicional y por creer en mí incluso en los momentos más difíciles.}
	
	\vspace{0.2cm}
	
	\textit{Y a mi cuatro patitas Duke, mi leal compañero, cuyo cariño y alegría hicieron más llevaderos los días de estudio.}
	
	\vspace{0.2cm}
	
	\textit{A todos ustedes, gracias por ser mi inspiración y mi fuerza.}
\end{flushright}
	% Agradecimientos
	%% Contenido de agradecimientos
\thispagestyle{plain}  % <- Muestra el número de página en esta hoja


\begin{center}
	{\bfseries\fontsize{13pt}{16pt}\selectfont AGRADECIMIENTOS}
\end{center}

\vspace{1em}

A la Universidad Mayor de San Andrés, y en especial a la Carrera de Informática de la Facultad de Ciencias Puras y Naturales, por proporcionar la formación académica que hizo posible la ejecución del presente proyecto.

Al Ph.D. Franz Cuevas Quiroz, tutor de este trabajo, por su guía constante, orientación metodológica y compromiso durante todas las etapas del desarrollo del proyecto.

A la Empresa de Transportes Cali Internacional, por permitir el acceso a información clave de sus procesos operativos, y por la disposición y colaboración del personal durante el levantamiento de requerimientos y la validación del sistema.

Asimismo, se agradece a todas las personas que brindaron apoyo técnico y académico en la implementación y evaluación del Software de logística y gestión de buses para transporte de pasajeros y envío de encomiendas.

A todos, mi más sincero reconocimiento y gratitud.
	% Resumen
	
	% Contenido del resumen
\chapter*{} % Deja el capítulo sin título aquí
\thispagestyle{plain}  % <- Muestra el número de página en esta hoja
\addcontentsline{toc}{chapter}{RESUMEN}

\begin{center}
	{\bfseries\fontsize{13pt}{16pt}\selectfont RESUMEN}
\end{center}

\vspace{1em}

El proyecto titulado “Software de logística y gestión de buses para transporte de pasajeros y envío de encomiendas” tiene como objetivo implementar un sistema computarizado que optimice los servicios de venta y reserva de pasajes, así como la gestión de la recepción y entrega de encomiendas en la empresa. Durante el análisis de procesos, se identificó que los registros de pasajes y encomiendas se realizaban de forma manual, lo que resultaba en un uso ineficiente del tiempo y los recursos materiales.

El desarrollo del sistema se fundamentó en un análisis detallado de las actividades de la empresa Cali Internacional, empleando el modelo en cascada como metodología principal. Los requisitos funcionales y no funcionales fueron obtenidos a través de observaciones directas y entrevistas con el personal. Para la implementación técnica, se seleccionó el framework Django de Python, aprovechando sus capacidades de programación orientada a objetos, junto con PostgreSQL como gestor de base de datos, debido a su excelente integración con Django.

La finalización del proyecto permitió una notable optimización de las operaciones, reflejada en una reducción del 64.62 por ciento en los tiempos de procesamiento. Las actividades relacionadas con el registro y la venta de boletos se agilizaron considerablemente, mejorando la atención al cliente y haciendo el proceso más eficiente. Este avance tuvo un impacto positivo en la productividad diaria y la calidad del servicio brindado.

\vspace{1em}
\noindent\textbf{Palabras clave:} pasajes, encomiendas, Python, Django.

	
	
	\clearpage
	\renewcommand{\contentsname}{\hfill\bfseries\normalsize CONTENIDO\hfill}
	\tableofcontents
	\thispagestyle{empty}
	\addtocontents{toc}{~\hfill\textbf{Pág.}\par}
	
	\addcontentsline{toc}{chapter}{RESUMEN} % Agregar al índice general
	
	% Índice de Figuras
	\listoffigures
	
	
	% Índice de Tablas
	\listoftables
	
	\mainmatter
	% Introduccción
	\setcounter{page}{1}
	\chapter{INTRODUCCIÓN}

	\vspace{10pt}

	Los avances actuales de la informática (2024) y la difusión global de la Internet han cambiado la manera en que se desarrollan las actividades de la sociedad en los ámbitos de la comunicación, la calidad de vida y el comercio. Internet ofrece nuevas alternativas de negocio ya que esta nos permite llegar a una audiencia masiva y a un gran número de posibles clientes; se puede ofrecer nuestros servicios a un mercado mucho mayor porque el tiempo y la distancia dejan de ser obstáculos \parencite{anormaliza2009implementacion}.
	
	% 16En esta era de la transformación digital, las Tecnologías de la Información y Comunicación (TICs) desempeñan un rol esencial al ser una combinación de servicios, redes, software y dispositivos completamente integrados. Las TICs se integran en un sistema de información interconectado y complementan un entorno económico y social, con el objetivo de mejorar constantemente las operaciones empresariales y la calidad de vida de los individuos. Las empresas y organizaciones utilizan las TICs con el objetivo principal de mejorar y acelerar los procesos internos, facilitar la toma de decisiones y obtener una ventaja competitiva notable en el mercado. Las organizaciones pueden mejorar su eficiencia y efectividad, así como destacarse en un entorno competitivo y siempre cambiante, gracias a la integración y uso estratégico de las TICs.
	
	Este desarrollo de la tecnología y su notable avance han hecho posible que los sistemas de información se integren en empresas, ya sean pequeñas, medianas o grandes. La competitividad del mercado ha sido el principal impulsor de este fenómeno, ya que obliga a las organizaciones a actualizar y mejorar sus mecanismos operativos para seguir siendo eficientes. Es fundamental en este escenario incorporar un sistema de información que no solo facilite la gestión y control de las operaciones, sino también brinde una solución completa para mejorar los procedimientos internos de la empresa. La adopción de estos sistemas tecnológicos brinda beneficios importantes al facilitar un seguimiento más preciso, la automatización de tareas repetitivas y una toma de decisiones mejorada mediante el uso de datos confiables en tiempo real. Además de mejorar la eficiencia operativa, esta acción también fortalece la capacidad de adaptación de la empresa a las demandas cambiantes del mercado.
	
	Según \textcite{casanueva2000practicas} una empresa es como una entidad que mediante la organización de elementos humanos, materiales, técnicos y financieros proporciona bienes o servicios a cambio de un precio que le permite la reposición de los recursos empleados y la consecución de unos objetivos determinados. Manejar grandes cantidades de información dentro de cualquier empresa demanda un nivel elevado de responsabilidad, usualmente, las compañías ponen más énfasis en la promoción de sus productos o servicios, sin embargo, es crucial no descuidar el aspecto administrativo.
	
	Para las empresas de transporte y logística la digitalización de sus servicios se ha convertido en un factor crucial para la competitividad y eficiencia, la integración de soluciones tecnológicas ha permitido a muchas organizaciones optimizar sus operaciones y mejorar la experiencia del cliente. Las empresas de transporte y logística enfrentan la necesidad de modernizar sus sistemas para satisfacer las expectativas de sus clientes. 
	
	En este contexto, el desarrollo de un software de logística y gestión de buses para transporte de pasajeros y envío de encomiendas representa una oportunidad significativa para modernizar las operaciones y mejorar la experiencia del cliente.
	
	Introducción Capítulo 1
	
	Introducción Capítulo 2
	
	Introducción Capítulo 3
	
	Introducción Capítulo 4
	
\section{ANTECEDENTES}

	La transformación digital ha impactado significativamente a diversas industrias, incluida la del transporte y la logística. La creciente demanda por servicios rápidos, eficientes y accesibles ha impulsado a las empresas a adoptar tecnologías avanzadas para mejorar sus operaciones.
	
	La implementación de software especializado en la venta de pasajes y gestión de encomiendas no es un concepto nuevo, pero su evolución ha sido notable. Con el tiempo, la incorporación de tecnologías más avanzadas, como bases de datos relacionales, interfaces de usuario mejoradas y capacidades de integración con otros sistemas, ha permitido el desarrollo de soluciones más robustas y eficientes. Estos avances han sido impulsados por la necesidad de mejorar la experiencia del cliente, reducir costos operativos y aumentar la competitividad en un mercado cada vez más exigente.
		
	% La adopción de tecnologías digitales ha permitido que empresas de transporte y logística modernicen sus procesos y aumenten su competitividad. Este fenómeno no se limita a grandes corporaciones, ya que incluso las empresas más pequeñas han comenzado a implementar soluciones tecnológicas que optimizan tanto la venta de boletos como la gestión de encomiendas. Estas herramientas no solo mejoran la eficiencia operativa, sino que también permiten una mayor precisión y rapidez en la atención al cliente. A medida que las demandas del mercado y de los usuarios evolucionan, la necesidad de soluciones flexibles y escalables ha impulsado la creación de sistemas de gestión que integran diversas funcionalidades, desde la planificación de rutas hasta el seguimiento de envíos. Esto garantiza una operación más controlada, ágil y transparente, alineada con las expectativas de los consumidores y los desafíos actuales del sector.
	
	La adopción de tecnologías como la computación en la nube, el Internet de las Cosas (IoT) y el análisis de big data ha abierto nuevas posibilidades para la gestión integral de operaciones en el transporte terrestre. Estas tecnologías permiten la creación de ecosistemas digitales donde la venta de boletos, la gestión de flotas y el manejo de encomiendas pueden integrarse de manera fluida y eficiente. Sin embargo, el desarrollo e implementación de tales sistemas integrales presenta desafíos significativos, desde la complejidad técnica hasta la necesidad de adaptarse a diversas regulaciones y prácticas operativas existentes.
	
	A nivel global, muchas empresas de transporte y logística han adoptado con éxito plataformas digitales para la venta de pasajes y gestión de encomiendas, logrando mejoras significativas en sus operaciones. Por ejemplo, compañías de renombre han implementado sistemas que permiten a los clientes reservar boletos y rastrear envíos en tiempo real, lo que ha aumentado la satisfacción del cliente y optimizado el flujo de trabajo interno.
	
	% El método en cascada, una metodología tradicional de desarrollo de software, se propone como el enfoque más adecuado para este proyecto debido a su estructura secuencial y sistemática. Este método permite una planificación y documentación detallada en cada etapa del desarrollo, asegurando que los requisitos del sistema se definan claramente desde el inicio. La naturaleza lineal del método en cascada facilita la gestión de grandes proyectos, permitiendo un seguimiento riguroso y la implementación de controles de calidad en cada fase. Dado que el proyecto implica la integración de múltiples funciones en una plataforma única, el método en cascada proporciona un marco sólido para garantizar que todas las partes del sistema se desarrollen y se integren de manera coherente.
	
	\subsection{Antecedentes institucionales}
	
	La empresa de transportes Cali Internacional, con sede en la Terminal de Buses de La Paz y número de NIT 491462023, es una compañía destacada en el sector del transporte y la logística en Bolivia, desde su fundación, Cali Internacional ha brindado servicios de venta de pasajes y gestión de encomiendas, ganándose una sólida reputación por su compromiso con la calidad y la satisfacción del cliente. La ubicación estratégica en la Terminal de Buses de La Paz permite a la empresa atender a un amplio espectro de clientes, facilitando tanto los viajes como el envío de paquetes de manera eficiente y segura.
	
	A lo largo de los años, Cali Internacional ha experimentado un crecimiento constante, adaptándose a los cambios del mercado y las necesidades de sus clientes. La empresa ha reconocido la importancia de incorporar tecnologías avanzadas para mejorar sus operaciones y mantenerse competitiva. Actualmente, enfrenta el desafío de modernizar sus procesos tradicionales de venta de pasajes y gestión de encomiendas, buscando una solución tecnológica que optimice sus operaciones y reduzca las ineficiencias. A continuación, en la (figura \ref{fig:figura1_1}) se muestra el organigrama de la empresa de transportes Cali Internacional.
	
	\vspace{0.3cm} % Agregar 1 cm de espacio entre el párrafo y la figura
	
	\begin{figure}[h] % 'H' del paquete 'float' para mantener posición	
		\caption[Descripción corta]
		{\newline Organigrama de la empresa de transportes ``Cali Internacional''.} % Leyenda en la parte superior
		\centering
		\includegraphics[width=0.55\textwidth]{imagenes/figura1_1.png} % Inserta una imagen
		
		\begin{flushleft}
			\hspace{1.20cm} \textbf{Nota.} Organigrama obtenido en entrevista con el administrador. % Nota al pie para esta figura
		\end{flushleft}
		\vspace{-16pt}
		\label{fig:figura1_1} % Etiqueta para referencia cruzada
	\end{figure}
	
	En la figura \ref{fig:figura1_1}, se observa la estructura organizativa de la empresa, destacando los diferentes cargos que desempeñan los empleados, que van desde el administrador hasta los auxiliares de apoyo. Dentro del negocio, se encuentran los boleteros y conductores, quienes son responsables de la atención directa a los pasajeros y la operación de los vehículos. Paralelamente, en la oficina central se gestionan las encomiendas, que son recibidas, clasificadas y preparadas para su envío. Cada uno de estos roles desempeña una función esencial para el funcionamiento eficiente y efectivo de la empresa, asegurando que tanto el transporte de pasajeros como la gestión de encomiendas se realicen con éxito y dentro de los estándares de calidad establecidos.
	
	\subsection*{Misión de la empresa}
	
	Proporcionar servicios de transporte y logística de alta calidad, enfocándose en la venta de pasajes y el envío de encomiendas, con el objetivo de satisfacer plenamente las necesidades de nuestros clientes. Nos comprometemos a ofrecer un servicio eficiente, seguro y confiable, contribuyendo al bienestar y comodidad de nuestros usuarios.
	
	\subsection*{Visión de la empresa}
	
	Ser la empresa líder en el sector del transporte y la logística en Bolivia, reconocida por nuestra innovación, eficiencia operativa y excelencia en el servicio al cliente. Aspiramos a expandir nuestra presencia y mejorar continuamente nuestros servicios para mantenernos a la vanguardia de la industria.
	
	\subsection*{Objetivo general de la empresa}
	El objetivo general de Cali Internacional es consolidar y expandir nuestra posición en el mercado del transporte y la logística, mejorando continuamente la calidad de nuestros servicios y adoptando tecnologías avanzadas para optimizar nuestras operaciones y satisfacer las necesidades cambiantes de nuestros clientes.
	
	\subsection*{Objetivos específicos de la empresa}
	
	\begin{itemize}[label=$\bullet$, left=0cm, labelsep = 1.05cm, topsep = 0pt, parsep = 0pt]
		
		\item Mejorar la experiencia del cliente mediante la oferta de servicios más rápidos, seguros y fiables.
		
		\item Capacitar continuamente a nuestro personal para asegurar que estén equipados con las habilidades necesarias para manejar las nuevas tecnologías y brindar un servicio de alta calidad.
		
		\item Implementar prácticas sostenibles en nuestras operaciones, minimizando el impacto ambiental y promoviendo la responsabilidad social corporativa.
		
	\end{itemize}	
	
	\subsection{Proyectos similares}
	
	Para la presente investigación se han considerado los siguientes antecedentes:
	
	\textcite{hurtado2019aplicacion}. ``Aplicación web administrativa para reserva de servicios de transporte y envío de encomiendas para la empresa Romero y Asociados (AMBASEUR) de la ciudad de Ambato''. En este proyecto, se implementó una aplicación web para automatizar los procesos manuales de una empresa, mejorando la gestión de reservas de transporte y envíos de encomiendas. La plataforma permite publicitar las actividades de la empresa y recopilar información precisa sobre los clientes. Desarrollada utilizando la metodología XP, la aplicación facilita la adaptación rápida a cambios y la incorporación de funciones adicionales, como un chat en línea, optimizando así la eficiencia y aumentando la base de clientes.
	
	\textcite{mora2022sistema}. ``Sistema gestión de servicio de viajes para la empresa Nuestra Señora de la Asunción C.I.S.A.'', esta investigación se centra en automatizar los procesos manuales de la empresa Nuestra Señora de la Asunción CISA mediante un sistema informático. En la primera etapa, se diagnosticaron los módulos de viajes, tráfico y ventas, entrevistando a responsables clave y recopilando los requerimientos necesarios. En la segunda etapa, se desarrolló un sistema informático web responsive que procesa automáticamente la información de estos módulos, integrando análisis, diseño y programación orientada a objetos, culminando en un sistema integrado con soporte audiovisual.
	
	\textcite{arevalo2021desarrollo}. ``Desarrollo de una aplicación web para
	agilizar los procesos de la compra y venta de boletos de buses interprovinciales en el terminal de Milagro.'', este proyecto desarrolló un sistema web para la compra y venta de boletos en el terminal terrestre del Cantón Milagro, con el objetivo de agilizar el proceso de boletería sin necesidad de contacto físico en ventanilla. Tras entrevistar a los socios del terminal para identificar los requisitos funcionales y no funcionales, se eligió la metodología ágil SCRUM para la organización y monitoreo constante del proyecto. El sistema se implementó utilizando Python, con Pycharm como IDE, Bootstrap 4 y Adminlte3 como plantillas, y PostgreSQL como base de datos. El resultado fue un sistema que satisface las necesidades del cliente, mejorando significativamente la experiencia de compra de boletos..
	
	\textcite{sosa2019sistema}. ``Sistema informático web para la gestión de pasajes de la empresa de transporte Turismo Transol Barranca S.A.C.'', en la tesis se propone como objetivo principal desarrollar un sistema informático web para la gestión de pasajes en la empresa de transportes Turismo Transol Barranca S.A.C., abarcando tanto la venta como la reserva de boletos. Este sistema busca optimizar el tiempo de procesamiento mediante el uso de tecnología web. La investigación se llevó a cabo con un enfoque descriptivo, un diseño no experimental y un corte transversal, utilizando una población de 42 personas y una muestra de 6 usuarios. Se aplicó la metodología Proceso Unificado de Rational (RUP), empleando el Lenguaje de Modelamiento Unificado (UML) para la construcción de diagramas de casos de uso, facilitando el análisis del software. El sistema fue desarrollado en Java, con MySQL como gestor de datos y MySQL Workbench 6.3 CE para el modelado de la base de datos, entre otras herramientas que ayudaron a cumplir los requisitos de diseño. Los resultados permitieron agilizar los procesos de venta y reserva de pasajes, mejorando el manejo de la información y extendiendo el alcance a los clientes, lo que fortaleció el posicionamiento competitivo de la empresa a nivel regional.
	
	\textcite{vivas2019propuesta}. ``Propuesta de implementación del sistema web de venta de boletos de viaje y gestión de encomiendas para la empresa Transportes Montero S.A.C. Piura; 2018.'', en esta investigación que fue desarrollada por la Escuela Profesional de Ingeniería de Sistemas de la Universidad Católica Los Ángeles de Chimbote, se centró en la mejora de procesos en organizaciones peruanas mediante la implementación de un sistema web para la venta de boletos y gestión de encomiendas en la empresa TRANSPORTES MONTERO S.A.C. en 2018. La investigación, de tipo cuantitativo y descriptivo con diseño no experimental y corte transversal, incluyó una muestra de 14 trabajadores. Los resultados mostraron que el 63 por ciento de los empleados consideraba que la empresa brindaba calidad en procesos y servicios, el 84 por ciento creía que los sistemas web agilizan los procesos, y el 81 por ciento opinaba que dichos sistemas eran eficientes, confirmando así la hipótesis planteada.
	
\section{OBJETO DE ESTUDIO}

	Software de logística y gestión de buses para transporte de pasajeros y envío de encomiendas, el cual va automatizar y mejorar la venta de pasajes, así como en la recepción, procesamiento y envío de encomiendas.
	
\section{PLANTEAMIENTO DEL PROBLEMA}

	La empresa de transportes Cali Internacional, se encuentra ante diversos retos importantes en cuanto a administrar sus procesos tanto de venta de pasajes como de envío de paquetería, estas operaciones son llevadas a cabo de forma manual, lo que genera ineficiencias en el funcionamiento, largos tiempos de espera para los clientes y una alta posibilidad de comoter errores. Además de afectar negativamente la experiencia del cliente, estos problemas también restringen las posibilidades de crecimiento y competencia efectiva en un mercado cada vez más digital, la implementación de soluciones tecnológicas integrales se ha convertido en una estrategia clave para optimizar procesos.  
	
	Algunos de los problemas mas frecuentes son:
	
	\begin{itemize}[label=$\bullet$, left=1.25cm, labelsep = 0.75cm, topsep = 0pt, parsep = 0pt]
		\item Largos tiempos de espera en la compra de pasajes y envío de encomiendas debido a la falta de automatización.
		\item Errores en la gestión de reservas y envíos, lo que puede resultar en pérdidas financieras y descontento entre los clientes.
		\item Falta de visibilidad y control sobre la demanda de servicios, lo que limita la capacidad de la empresa para ajustar su oferta y optimizar recursos.
		\item Dificultad para generar reportes y análisis que ayuden en la toma de decisiones estratégicas para la empresa.    
	\end{itemize}
	
	Por lo tanto, se plantea la siguiente interrogante:
	
	?`Cómo mejorar la venta de pasajes y la gestión de envío de encomiendas en la empresa Cali Internacional?
	
\section{JUSTIFICACIÓN}

	La implementación de un software de logística y gestión de buses para transporte de pasajeros y envío de encomiendas representa una respuesta estratégica ante la creciente demanda de soluciones tecnológicas en el sector de transporte y logística. La automatización de estos procesos no solo optimiza las operaciones internas, sino que también reduce significativamente los errores humanos y mejora la eficiencia. En vista de ello la mayoría de organizaciones se ha visto obligada a desarrollar un sistema web de calidad que brinde un mejor servicio a la comunidad, mejorando su imagen corporativa, demostrando que están al día con las nuevas tecnologías \parencite{nunez2005diseno}.
	
	Además, este proyecto aborda la necesidad de ofrecer un servicio más accesible y conveniente para los clientes. En un entorno donde la digitalización se ha vuelto imprescindible, la adopción de un sistema informático para estos servicios es una ventaja competitiva que no se puede ignorar.
	
	La digitalización de estos procesos en una plataforma única no solo agilizará las operaciones al automatizar tareas repetitivas y reducir la necesidad de intervención manual, sino que también mejorará significativamente la precisión y la transparencia de la información. Esta mejora permitirá a la empresa ofrecer un servicio más coherente y eficiente, ya que todos los datos estarán centralizados y accesibles en tiempo real, lo que facilitará una gestión más efectiva de los recursos. Además, la integración de estos procesos en una sola plataforma reducirá costos operativos al eliminar redundancias y optimizar el uso de la infraestructura tecnológica. En última instancia, esto resultará en una mejor experiencia para el cliente, aumentando su satisfacción al recibir un servicio más rápido y confiable, y posicionando a la empresa como líder en innovación y eficiencia dentro de su sector.
	
	Este proyecto se adapta a la necesidad de mantenerse al día con las tendencias tecnológicas actuales. Las empresas están siendo revolucionadas por la transformación digital, y aquellas que no se adapten corren el riesgo de quedarse atrás. Cuando la empresa implementa un software especializado, no solo se adapta a estas tendencias, sino que también está preparada para hacer frente a los desafíos futuros como la necesidad de incorporar nuevas tecnologías y responder a las demandas del mercado en constante cambio.
	
\section{OBJETIVOS}
	\subsection{Objetivo general}
	
		Desarrollar un software de logística y gestión de buses para transporte de pasajeros y envío de encomiendas para la empresa de transportes Cali Internacional de la ciudad de La Paz.
		
	\subsection{Objetivos específicos}
	
		\begin{itemize}[label=$\bullet$, left=0cm, labelsep = 1.05cm, topsep = 0pt, parsep = 0pt]
			
			\item Analizar los procesos actuales de venta de pasajes y envío de encomiendas en la empresa de transportes Cali Internacional para identificar las áreas de mejora y las necesidades tecnológicas específicas.
			\item Diseñar una interfaz de usuario intuitiva y accesible que permita a los empleados interactuar con el sistema de manera sencilla, facilitando la usabilidad del software.
			\item Elaborar el diseño de la base de datos relacional a partir del análisis de los requerimientos del sistema, para llevar a la Tercera Forma Normal (3FN) y almacenar los datos.
			\item Diseñar el back-end para gestionar la venta de pasajes y el envío de encomiendas, asegurando la integración eficiente con la base de datos y la correcta ejecución de las operaciones solicitadas por los usuarios a través de la plataforma digital.    
			\item Generar reportes y análisis de datos que facilite la toma de decisiones informadas por parte de la administración de la empresa.
			
		\end{itemize}
		
\section{ALCANCES Y LÍMITES}
	\subsection{Alcances}
		
		El desarrollo de la presente investigación se encuentra dentro de los siguientes alcances:
		
		\begin{itemize}[label=$\bullet$, left=0cm, labelsep = 1.05cm, topsep = 0pt, parsep = 0pt]
			
			\item El proyecto abarcará la creación de una plataforma digital que permita a los usuarios realizar la compra de pasajes y la gestión de envíos de encomiendas de manera eficiente y segura.
			
			\item Se desarrollará un sistema de gestión de usuarios que permitirá a los empleados: iniciar sesión y gestionar las ventas de pasajes y envíos de encomiendas, mientras que los administradores podrán supervisar y manejar las operaciones.
			
			\item Se implementarán módulos que automatizarán tareas repetitivas como la generación de recibos, el seguimiento de envíos y la asignación de asientos en los buses de transportes.
			
			\item El sistema incluirá un módulo de reportes que permitirá a los administradores generar informes detallados sobre las ventas, la ocupación de los transportes, y la gestión de encomiendas.
			
			\item La plataforma será accesible desde diferentes tipos de dispositivos, incluyendo computadoras, tabletas, y smartphones, garantizando una experiencia de usuario consistente y accesible.
			
		\end{itemize}
		
	\subsection{Límites}
	
		Los límites de la investigación son los siguientes:
		
		\begin{itemize}[label=$\bullet$, left=0cm, labelsep = 1.05cm, topsep = 0pt, parsep = 0pt]
			
			\item El sistema estará diseñado inicialmente para cubrir las operaciones de la Empresa de transportes Cali Internacional en su sede de la Terminal de Buses en La Paz.
			
			\item La integración se centrará en los sistemas internos existentes de la empresa. %dejando de lado conexiones con plataformas o sistemas externos.
			
			\item El software será compatible con las plataformas y dispositivos especificados. %sin incluir soporte para otros sistemas no contemplados inicialmente.
			
			\item El soporte se limitará a las funcionalidades implementadas, las actualizaciones o desarrollos adicionales quedarán para fases futuras.
			
		\end{itemize}
		
\section{IMPORTANCIA DEL ESTUDIO}

	La importancia del estudio del proyecto radica en la necesidad de modernizar los procesos operativos de empresas de transporte y logística, especialmente en un entorno donde la eficiencia y la rapidez son factores clave para la competitividad. En la actualidad, muchas empresas en este sector aún dependen de sistemas manuales o desactualizados que ralentizan las operaciones, sino que también incrementan el riesgo de errores humanos, afectando directamente la calidad del servicio ofrecido al cliente. Este proyecto, por lo tanto, no solo aborda una necesidad tecnológica, sino que también busca mejorar la experiencia del cliente al ofrecerle un servicio más ágil y fiable.
	
	Desde una perspectiva social, este estudio tiene una importancia significativa al contribuir al avance tecnológico en un sector que afecta directamente a un gran número de personas. Al mejorar la eficiencia y la precisión en la venta de pasajes y el envío de encomiendas, se generan beneficios directos no solo para la empresa, sino también para los usuarios finales, quienes experimentarán un servicio más confiable y accesible. Esto, a su vez, puede fomentar una mayor confianza en los servicios digitales en general, impulsando el uso de la tecnología en otras áreas de la vida diaria.
	
	Finalmente, la importancia de este estudio también reside en su capacidad para servir como modelo para futuras implementaciones tecnológicas en empresas similares. La metodología empleada, así como los desafíos superados durante el desarrollo del software, pueden ofrecer valiosas lecciones para otros proyectos dentro del sector, promoviendo un enfoque más sistemático y eficiente en la adopción de tecnologías de la información en la industria del transporte y la logística.
	
	% Marco teórico
	\chapter{MARCO TEÓRICO} 
	
	\vspace{0pt}
	
	\section{LOGÍSTICA DEL TRANSPORTE DE PASAJEROS}
	\subsection{Logística}
		``Logística es planificar, operar, controlar y detectar oportunidades de mejora del proceso de flujo de materiales (insumos, productos), servicios, información y dinero.  Es la función que normalmente opera como nexo entre las fuentes de aporvisionamiento y suministro y el cliente final o la distribución.  Su objetivo es satisfacer permanentemente la demanda en cuanto a cantidad, oportunidad y calidad al menor costo posible para la empresa.''\parencite{carro2013logistica}
	\subsection{Transporte}
	
	\subsection{Sistema de transporte}
	\subsection{Funcionalidad del transporte}
	\section{RESERVA Y VENTA DE PASAJES}
	\subsection{Proceso de reserva y venta de pasajes}
	\subsection{Emisión y gestion de boletos}
	\subsection{Cambios y cancelaciones}
	\section{SERVICIO DE TRANSPORTE DE PASAJEROS Y ENCOMIENDAS}
	\section{LOGÍSTICA DE ENVÍO DE ENCOMIENDAS}
	\subsection{Encomienda}
	\subsection{Proceso de recepción}
	\subsection{Clasificación de encomiendas}
	\subsection{Embalaje y etiquetado}
	\subsection{Entrega al destinatario}
	\section{GESTIÓN DE BUSES}
	\subsection{Asignación de rutas}
	\subsection{Análisis y demanda de pasajeros}
	\subsection{Programación y asignación de conductores}
	
	
		
	%\vspace{0.3cm} % Agregar 1 cm de espacio entre el párrafo y la figura
		
	%	\begin{figure}[h] % 'H' del paquete 'float' para mantener posición	
	%		\caption[Descripción corta]
	%		{\newline Resultados del informe PISA 2022.} % Leyenda en la parte superior
	%		\centering
	%		\includegraphics[width=0.95\textwidth]{imagenes/figura2_1.png} % Inserta una imagen
	%		
	%		\begin{flushleft}
	%			\hspace{1.20cm} \textit{Nota.} al pie asociada con esta figura, explicando detalles adicionales. % Nota al pie para esta figura
	%		\end{flushleft}
	%		\vspace{-16pt}
	%		\label{fig:figura2_1} % Etiqueta para referencia cruzada
	%	\end{figure}

	
	
	\section{MARCO LEGAL Y NORMATIVO}
	\subsection{Autoridad de regulación y fiscalización de telecomunicaciones y transportes}
	\subsection{Ley general del transporte}
	\subsection{Reglamento regulatorio de transporte terrestre}



	% Marco aplicativo
	\chapter{MARCO APLICATIVO}
\section{ANÁLISIS Y DEFINICIÓN DE REQUERIMIENTOS}
	En la primera etapa del desarrollo del software, se realiza el Análisis y definición de requerimientos, una fase esencial donde se identifican y documentan las necesidades y expectativas del cliente mediante la ingeniería de requerimientos. Esta etapa es importante para asegurar que el desarrollo posterior esté alineado con las metas del proyecto y que el sistema cumpla efectivamente con las expectativas planteadas, evitando problemas en fases avanzadas del desarrollo.
	
	\subsection{Levantamiento de requerimientos}
	Para la realización del levantamiento de requerimientos se realizó un cronograma de reuniones que se observa en la tabla \ref*{tab:tabla3_1}.
	\vspace{-1pt}  % O el valor que necesites para ajustar
	% Nota personalizada fuera de `\caption*{}`
	% \textbf{Nota}: Esta es la nota de la tabla, explicando datos relevantes.
	
	\begin{longtable}{>{\centering\arraybackslash}m{2cm} >{\centering\arraybackslash}m{3cm} >{\centering\arraybackslash}m{9.5cm}}
		\caption[Cronograma de reuniones y entrevistas]{\newline Cronograma de reuniones y entrevistas} \label{tab:tabla3_1}\\
		\toprule
		\textbf{Reunión} & \textbf{Fecha} & \textbf{Tarea}\\
		\midrule
		\endfirsthead
		
		\toprule
		\textbf{Reunión} & \textbf{Fecha} & \textbf{Tarea}\\
		\midrule
		\endhead
		
		%\midrule
		%\multicolumn{3}{r}{\textit{Continúa en la siguiente página}} \\
		%\midrule
		%\endfoot
		
		\bottomrule
		\endlastfoot
		
		% Aquí se colocan las filas de la tabla, por ejemplo:
		1 & 09/12/2024 & Reunión con el Administrador \\
		2 & 03/01/2025 & Entrevista con el Administrador \\
		3 & 07/02/2025 & Entrevista con el Personal de Atención al Cliente \\
			
	\end{longtable}
	\vspace{-12pt}  % O el valor que necesites para ajustar
	% Nota personalizada fuera de `\caption*{}`
	\textbf{Nota}: El cronograma muestra las fechas para las reuniones y entrevistas realizadas durante el desarrollo del proyecto.
	
	Como parte inicial al proceso de levantamiento de requerimientos se realizó la \textbf{técnica de observación directa} sobre las operaciones diarias en la empresa de transportes durante la semana del 16 al 20 de diciembre de 2024. Esta técnica permitió identificar puntos críticos iniciales para preparar las entrevistas posteriores al personal de la empresa.
	
	Durante la observación se identificaron los siguientes aspectos:
	
	\begin{enumerate}[left=0.1cm, labelsep = 0.9cm, topsep = 0pt, parsep = 0pt]
		\item Procesos que requieren optimización:
		\begin{itemize}[label=$-$, left=0cm, labelsep = 0.9cm, topsep = 0pt, parsep = 0pt]
			\item Verificación de disponibilidad de asientos
			\item Registro de encomiendas
			\item Control de embarque de pasajeros
			\item Gestión de caja y turnos
		\end{itemize}		
		\item Puntos críticos identificados:
		\begin{itemize}[label=$-$, left=0cm, labelsep = 0.9cm, topsep = 0pt, parsep = 0pt]
			\item Tiempos de espera prolongados en horas pico
			\item Proceso manual propenso a errores
			\item Falta de información en tiempo real
		\end{itemize}
		\item Oportunidades de mejora:
		\begin{itemize}[label=$-$, left=0cm, labelsep = 0.9cm, topsep = 0pt, parsep = 0pt]
			\item Automatización del proceso de venta
			\item Gestión digital de asientos
			\item Control automatizado de embarque
		\end{itemize}
	\end{enumerate}
		
	Las entrevistas fueron programadas en base al cronograma detallado en la tabla \ref{tab:tabla3_1}, logrando reuniones efectivas tanto con el administrador y con el personal de atención al cliente. A continuación, se detallan las consultas realizadas en estas sesiones.
		
	\textbf{Entrevista al Administrador}
	
	Información del Entrevistado
	\begin{itemize}[label=$-$, left=0cm, labelsep = 0.9cm, topsep = 0pt, parsep = 0pt]
		\item Cargo: Administrador General
		\item Fecha: 03/01/2025
		\item Lugar: Empresa de Transportes Cali Internacional
		\item Duración estimada: 60 minutos
	\end{itemize}
	
	Preguntas:
	
	\begin{enumerate}[left=0.1cm, labelsep = 0.9cm, topsep = 0pt, parsep = 0pt]
		\item ¿Cuáles son los principales problemas o dificultades que enfrenta la empresa con el manejo actual?
		\item ¿Qué procesos considera que son los más críticos y necesitan mayor atención?
		\item ¿Cómo se gestiona actualmente el control de buses y encomiendas en la empresa?
		\item ¿Cuántos buses tiene actualmente la empresa en operación?
		\item ¿Cuántas rutas manejan y cuáles son sus principales destinos?
		\item ¿Qué problemas son los más frecuentes en el servicio de encomiendas?
		\item ¿Qué volumen promedio de encomiendas manejan diariamente?
		\item ¿Qué funcionalidades específicas necesita que tenga el módulo de gestión de buses para optimizar las operaciones?
		\item ¿Cómo se realiza actualmente la asignación de conductores a las unidades?
		\item ¿Cuántos empleados necesitarán acceso al sistema?
		\item ¿Qué roles o niveles de acceso considera necesarios para el personal?
		\item ¿Qué tipos de reportes son esenciales para la toma de decisiones?
		\item ¿Cómo le gustaría que se maneje el sistema de reservas y venta de pasajes?
		\item ¿Qué sistema de tarifas manejan y cómo les gustaría que se implemente en el software?
		\item ¿Qué tipo de reportes financieros necesita generar periódicamente?
		\item ¿En qué plazo espera que el sistema esté completamente operativo?
	\end{enumerate}
	
	\textbf{Entrevista al Personal de atención al cliente}
		
	Información del Entrevistado
	\begin{itemize}[label=$-$, left=0cm, labelsep = 0.9cm, topsep = 0pt, parsep = 0pt]
		\item Vendedor de pasajes y recepcionista de encomiendas
		\item Fecha: 07/02/2025
		\item Lugar: Empresa de Transportes Cali Internacional
		\item Duración estimada: 60 minutos
	\end{itemize}
	
	Preguntas:
	
	\begin{enumerate}[left=0.1cm, labelsep = 0.9cm, topsep = 0pt, parsep = 0pt]
		\item ¿Cuál es el proceso actual que sigue para vender un boleto de viaje?
		\item ¿Qué información del cliente es obligatoria registrar al momento de la venta?
		\item ¿Cómo maneja las reservaciones de asientos?
		\item ¿Qué problemas son los más frecuentes durante el proceso de venta de boletos?
		\item ¿Cómo gestiona actualmente los diferentes tipos de tarifas?
		\item ¿Cuál es el procedimiento actual para registrar una encomienda?
		\item ¿Qué información necesita registrar sobre las encomiendas?
		\item ¿Cómo realiza el seguimiento de una encomienda cuando un cliente lo solicita?
		\item ¿Qué problemas son los más comunes en el servicio de encomiendas?
		\item ¿Cómo maneja las quejas por pérdida o retraso de encomiendas?
		\item ¿Cuáles son las preguntas más frecuentes de los clientes?
		\item ¿Qué información necesita tener a mano para responder rápidamente a las consultas de los clientes?
		\item ¿Cómo gestiona los cambios o cancelaciones de pasajes?
		\item ¿Cómo realiza el cierre de caja de sus ventas?
		\item ¿Qué tipo de reportes necesita generar durante su turno?
		\item ¿Cómo verifica la disponibilidad de asientos en los buses?
		\item ¿Qué reportes le facilitarían su trabajo diario?
		\item ¿Qué proceso sigue cuando un cliente pierde su boleto?
		\item ¿Qué experiencia tiene en el uso de sistemas informáticos?
		\item ¿Qué aspectos considera importantes incluir en la capacitación del nuevo sistema?
		\item ¿Qué información proporciona a los clientes sobre el viaje?
	\end{enumerate}
	
	Las entrevistas permiten complementar la información obtenida en la observación directa y proporcionar una visión más clara de los procesos actuales y las necesidades reales de automatización.	
	
	\subsection{Análisis de requerimientos}
	Durante la fase de Análisis de requerimientos del proyecto, se realizó un proceso integro para definir y estructurar los casos de uso que abarcarán las funcionalidades esenciales del sistema, esta etapa se centró en organizar las interacciones clave que los usuarios tendrán con el sistema, permitiendo establecer una visión clara de cómo deben funcionar los distintos módulos.
	
	En esta fase se realizó la organización y documentación de los casos de uso, lo que facilitó el establecimiento de una visión clara de las funcionalidades a desarrollar, se logró crear una especificación detallada de cada caso de uso, incluyendo actores, flujos principales, flujos alternativos y condiciones específicas de ejecución. Este nivel de detalle proporciona una guía clara para las fases subsecuentes del proyecto, también permite establecer un entendimiento común con los interesados sobre cómo el sistema debe comportarse ante las diferentes interacciones de los usuarios.\\
	\\
	\textbf{Diagrama de casos de uso}
		
	De acuerdo al análisis de requerimientos en la figura \ref{fig:caso_uso}, se presenta el Diagrama de casos de uso para el sistema.
	
	\begin{figure}[!h] % 'H' del paquete 'float' para mantener posición	
		\caption[Diagrama de Casos de Uso]
		{\newline Diagrama de Casos de Uso del sistema de logística y gestión de buses.} % Leyenda en la parte superior
		\vspace{0.3cm}
		\centering
		\includegraphics[width=0.9\textwidth]{imagenes/cap_3/casos_de_uso.png} % Inserta una imagen
		\vspace{0.3cm}
		\caption*{\textup{\textbf{Nota}: El diagrama representa los principales actores y casos de uso del sistema.}}
		\vspace{-0.8cm}
		\label{fig:caso_uso} % Etiqueta para referencia cruzada
	\end{figure}
	
	\noindent \textbf{Especificaciones de casos de uso}
	
	El presente apartado desde la tabla \ref{tab:tabla3_2} a la tabla \ref{tab:tabla3_10} contiene la especificación detallada de los casos de uso del sistema, organizados por módulos funcionales, para cada caso se describe el alcance, los objetivos específicos y las interacciones necesarias para completar las funcionalidades requeridas, proporcionando una visión clara del comportamiento esperado.
	
	\begingroup
	% \onehalfspacing
	
	\begin{longtable}{m{7.5cm}|m{7.5cm}}
		\caption[Especificación de casos de uso: Autenticación del usuario]{\newline Especificación de casos de uso: Autenticación del usuario} \label{tab:tabla3_2}\\
		\toprule
		\textbf{Caso de Uso:} Autenticación del usuario & \textbf{Actores:} Administrador, Personal de atención al cliente \\
		\midrule
		\endfirsthead
		
		\textbf{Caso de Uso:} Autenticación del usuario & \textbf{Actores:} Administrador, Personal de atención al cliente \\
		\midrule
		\endhead
		
		\bottomrule
		\endlastfoot
		
		\multicolumn{2}{m{15cm}}{\textbf{Descripción:} La página de autenticación de usuarios permite al administrador y al personal de atención al cliente acceder al sistema y realizar funciones de cada usuario.} \\ \hline
		
		\multicolumn{2}{m{15cm}}{\textbf{Secuencia Normal:}

			El usuario y la contraseña son validados en la base de datos.
			
			Se verifica en la base de datos el tipo de usuario que se ha autenticado y se lo dirige a las opciones pertinentes.
			
			Se visualizan las opciones que tiene cada usuario.
		} \\ \hline
		
		\textbf{Precondiciones:} El usuario debe contar con un nombre de usuario y una contraseña para poder acceder a las opciones del sistema web. & \textbf{Postcondiciones:} Los datos del usuario se mantienen mientras su sesión esté abierta después de que se ha autenticado en el sistema. \\ \hline
		
		\multicolumn{2}{m{15cm}}{\textbf{Excepciones:} Si el usuario y la contraseña no existen en la base de datos o si la contraseña no corresponde al usuario, se muestra una notificación de error.
		} \\
		
	\end{longtable}
	
	\endgroup
	 
	\vspace{-6pt}  % O el valor que necesites para ajustar
	% Nota personalizada fuera de `\caption*{}`
	\textbf{Nota}: Este caso de uso describe el proceso de autenticación de los usuarios.
	
\begingroup
% \onehalfspacing

	\begin{longtable}{m{7.5cm}|m{7.5cm}}
		\caption[Especificación de casos de uso: Gestionar bus]{\newline Especificación de casos de uso: Gestionar bus} \label{tab:tabla3_3}\\
		\toprule
		\textbf{Caso de Uso:} Gestionar bus & \textbf{Actores:} Administrador \\
		\midrule
		\endfirsthead
	
		\textbf{Caso de Uso:} Gestionar bus & \textbf{Actores:} Administrador \\
		\midrule
		\endhead
	
		\bottomrule
		\endlastfoot
	
		\multicolumn{2}{m{15cm}}{\textbf{Descripción:} Este caso de uso hace referencia al registro de los datos de un bus.} \\ \hline
	
		\multicolumn{2}{m{15cm}}{\textbf{Secuencia Normal:}
		
			El administrador debe elegir la opción “Configurar Bus” del ítem “Configurar”.
		
			El administrador debe consultar la existencia del bus a registrar.
		
			El administrador debe completar la placa del bus, la capacidad de personas, modelo y año.
		
			El administrador guarda el registro realizado.
		
			El administrador podrá editar los datos del registro y eliminar el registro realizado.
		} \\ \hline
	
		\textbf{Precondiciones:} El administrador debe acceder al sistema con su usuario y contraseña. & \textbf{Postcondiciones:} Ninguna. \\ \hline
	
		\multicolumn{2}{m{15cm}}{\textbf{Excepciones:}
		
			Si el usuario y la contraseña no existen en la base de datos o si la contraseña no corresponde al usuario, se muestra una notificación de error solicitando nuevamente los datos.
		
			De no completar los datos en los registros, se mostrará un mensaje mencionando qué datos están vacíos y cuáles no han sido seleccionados.
		} \\
	\end{longtable}
	\endgroup
	\vspace{-6pt}  % O el valor que necesites para ajustar
	% Nota personalizada fuera de `\caption*{}`
	\textbf{Nota}: El caso de uso cubre el proceso de registro de un bus en el sistema, incluyendo los detalles como la placa, capacidad, modelo y año, aso como la opción de editar o eliminar estos datos.

	
	\begingroup
	% \onehalfspacing
	
	\begin{longtable}{m{7.5cm}|m{7.5cm}}
		\caption[Especificación de casos de uso: Gestionar rutas]{\newline Especificación de casos de uso: Gestionar rutas} \label{tab:tabla3_4}\\
		\toprule
		\textbf{Caso de Uso:} Gestionar rutas & \textbf{Actores:} Administrador \\
		\midrule
		\endfirsthead
		
		\textbf{Caso de Uso:} Gestionar rutas & \textbf{Actores:} Administrador \\
		\midrule
		\endhead
		
		\bottomrule
		\endlastfoot
		
		\multicolumn{2}{m{15cm}}{\textbf{Descripción:} Este caso de uso hace referencia al registro de los lugares de origen y destino.} \\ \hline
		
		\multicolumn{2}{m{15cm}}{\textbf{Secuencia Normal:}
			
			El administrador debe elegir la opción “Registrar rutas” del ítem “Registro”.
			
			El administrador debe consultar la existencia de las rutas de origen y de destino a registrar.
			
			El administrador debe completar los datos del lugar de origen como el nombre de la ciudad. Por defecto el estado muestra como activo. De igual manera, el administrador deberá completar los mismos datos para el lugar de destino.
			
			El administrador guarda el registro realizado.
			
			El administrador podrá editar los datos del registro, eliminar el registro realizado y cambiar el estado del registro.
		} \\ \hline
		
		\textbf{Precondiciones:} El administrador debe acceder al sistema con su usuario y contraseña. & \textbf{Postcondiciones:} Ninguna. \\ \hline
		
		\multicolumn{2}{m{15cm}}{\textbf{Excepciones:}
			
			Si el usuario y la contraseña no existen en la base de datos o si la contraseña no corresponde al usuario, se muestra una notificación de error solicitando nuevamente los datos.
			
			De no completar los datos en los registros, se mostrará un mensaje mencionando qué datos están vacíos y cuáles no han sido seleccionados.
		} \\
	\end{longtable}
	\endgroup
	\vspace{-6pt}  % O el valor que necesites para ajustar
	% Nota personalizada fuera de `\caption*{}`
	\textbf{Nota}: Este caso de uso se refiere al registro de rutas, en el cual el administrador debe ingresar los lugares de origen y destino, con la opción de editar o eliminar estos registros posteriormente.	
	
	\begingroup
	% \onehalfspacing	
	\begin{longtable}{m{7.5cm}|m{7.5cm}}
		\caption[Especificación de casos de uso: Gestionar viajes]{\newline Especificación de casos de uso: Gestionar viajes} \label{tab:tabla3_5}\\
		\toprule
		\textbf{Caso de Uso:} Gestionar viajes & \textbf{Actores:} Administrador \\
		\midrule
		\endfirsthead
		
		\textbf{Caso de Uso:} Gestionar viajes & \textbf{Actores:} Administrador \\
		\midrule
		\endhead
		
		\bottomrule
		\endlastfoot
		
		\multicolumn{2}{m{15cm}}{\textbf{Descripción:} Este caso de uso hace referencia a la programación de los viajes con sus fechas respectivas para cada uno. Se asignará un viaje a un bus y, por defecto, se indicarán los lugares de origen y destino. Tendrá un estado de viaje.} \\ \hline
		
		\multicolumn{2}{m{15cm}}{\textbf{Secuencia Normal:}
			
			El administrador debe elegir la opción “Registrar viajes” del ítem “Registro”.
			
			El administrador debe consultar el bus que realizará un viaje.
			
			El sistema muestra el lugar de ubicación del bus y por defecto lo asigna al campo de ciudad de origen. También se completan los campos de capacidad de personas y límite de carga.
			
			El administrador debe completar la ciudad de destino, debe programar la fecha de salida y de llegada del viaje. Debe elegir un estado del viaje que, por defecto, el sistema muestra como PENDIENTE.
			
			El administrador guarda el registro realizado.
			
			El administrador podrá editar los datos del registro y eliminar el registro realizado.
		} \\ \hline
		
		\textbf{Precondiciones:} El administrador debe acceder al sistema con su usuario y contraseña. & \textbf{Postcondiciones:} Ninguna. \\ \hline
		
		\multicolumn{2}{m{15cm}}{\textbf{Excepciones:} De no completar los datos en los registros, se mostrará un mensaje mencionando qué datos están vacíos y cuáles no han sido seleccionados.
		} \\
	\end{longtable}
	\endgroup
	\vspace{-6pt}  % O el valor que necesites para ajustar
	% Nota personalizada fuera de `\caption*{}`
	\textbf{Nota}: Este caso de uso trata sobre la programación de viajes, donde el administrador asigna un bus y establece la fecha y hora de salida.
	
	\begingroup
	% \onehalfspacing
	\begin{longtable}{m{7.5cm}|m{7.5cm}}
		\caption[Especificación de casos de uso: Gestionar venta y reserva de pasaje]{\newline Especificación de casos de uso: Gestionar pasaje} \label{tab:tabla3_6}\\
		\toprule
		\textbf{Caso de Uso:} Gestionar venta y reserva de pasajes & \textbf{Actores:} Personal de atención al cliente \\
		\midrule
		\endfirsthead
		\endhead
		\bottomrule
		\endlastfoot
		\multicolumn{2}{m{15cm}}{\textbf{Descripción:} Este caso de uso hace referencia a la venta y reservas de pasajes para los buses.} \\ \hline
		
		\multicolumn{2}{m{15cm}}{\textbf{Secuencia Normal:}
			
			El vendedor debe elegir la opción “Pasajes”.
			
			El vendedor consulta el día del viaje, el sistema mostrará la lista de viajes programados según la fecha actual del sistema. Por defecto se completan los datos de la ciudad de origen y de destino y el bus programado.
			
			El vendedor debe consultar los asientos libres en los buses disponibles.
			
			El vendedor consulta la existencia del cliente; si existe, se obtienen los datos y se completan en los campos vacíos; si no existe, procede a registrar los datos del cliente.
			
			Por defecto se completan los datos en los demás campos según el registro de los datos de las personas. El sistema consulta, calcula el monto del pago y completa los campos.
			
			Se imprime el boleto del pasaje con los datos del pasajero, el destino, el monto de pago, el tipo de pasaje y un número de ticket.
			
			El vendedor también podrá editar los datos del registro del pasaje y cancelar el registro.
		} \\ \hline
		
		\textbf{Precondiciones:} El vendedor debe acceder al sistema con su usuario y contraseña. & \textbf{Postcondiciones:} Ninguna. \\ \hline
		
		\multicolumn{2}{m{15cm}}{\textbf{Excepciones:} De no completar los datos en los registros, se mostrará un mensaje mencionando qué datos están vacíos y cuáles no han sido seleccionados.
		} \\
		
	\end{longtable}
	\endgroup 
	\vspace{-20pt}  % O el valor que necesites para ajustar
	% Nota personalizada fuera de `\caption*{}`
	\textbf{Nota}: Este caso de uso describe el proceso de venta y reserve de pasajes, incluyendo la consulta de disponibilidad, registro de datos del cliente y emisión del boleto.
	
	\begingroup
	% \onehalfspacing
	
	\begin{longtable}{m{7.5cm}|m{7.5cm}}
		\caption[Especificación de casos de uso: Gestionar rol de usuario]{\newline Especificación de casos de uso: Gestionar rol de usuario} \label{tab:tabla3_9}\\
		\toprule
		\textbf{Caso de Uso:} Gestionar rol de usuario & \textbf{Actores:} Administrador \\
		\midrule
		\endfirsthead
		\endhead
		\bottomrule
		\endlastfoot
		
		\multicolumn{2}{m{15cm}}{\textbf{Descripción:} Este caso de uso hace referencia al registro de un rol de usuario.} \\ \hline
		
		\multicolumn{2}{m{15cm}}{\textbf{Secuencia Normal:}
			
			El administrador debe elegir la opción “Configurar roles”.
			
			El administrador debe consultar la existencia del rol a registrar.
			
			El administrador debe completar los campos de nombre del rol y registrar una descripción del mismo.
			
			El administrador debe seleccionar los accesos que tendrá el rol al sistema.
			
			El administrador podrá editar los datos del registro y eliminar el registro realizado.
		} \\ \hline
		
		\textbf{Precondiciones:} El administrador debe acceder al sistema con su usuario y contraseña. & \textbf{Postcondiciones:} Ninguna. \\ \hline
		
		\multicolumn{2}{m{15cm}}{\textbf{Excepciones:} De no completar los datos en los registros, se mostrará un mensaje mencionando qué datos están vacíos y cuáles no han sido seleccionados.
		} \\
	\end{longtable}
	\endgroup
	\vspace{-6pt}  % O el valor que necesites para ajustar
	% Nota personalizada fuera de `\caption*{}`
	\textbf{Nota}: Caso de uso que describe el proceso de asignación de roles a los usuarios del sistema.
		
	\begingroup
	% \onehalfspacing
	\begin{longtable}{m{7.5cm}|m{7.5cm}}
		\caption[Especificación de casos de uso: Gestionar encomienda]{\newline Especificación de casos de uso: Gestionar encomienda} \label{tab:tabla3_8}\\
		\toprule
		\textbf{Caso de Uso:} Gestionar encomienda & \textbf{Actores:} Personal de atención al cliente \\
		\midrule
		\endfirsthead
		\endhead
		\bottomrule
		\endlastfoot
		\multicolumn{2}{m{15cm}}{\textbf{Descripción:} Este caso de uso hace referencia al registro de las encomiendas.} \\ \hline
		
		\multicolumn{2}{m{15cm}}{\textbf{Secuencia Normal:}
			
			El recepcionista debe elegir la opción “Enviar encomienda”.
			
			El recepcionista debe consultar el día del viaje. Por defecto se completan los campos del lugar de origen y destino y el bus programado.
			
			El recepcionista realiza una consulta de la existencia de los clientes (remitente y destinatario); si existen se obtienen los datos y se completan los campos vacíos. Si no existen, se procede a registrar sus datos.
			
			El recepcionista debe registrar los datos de la encomienda o carga, llenar una descripción de la encomienda, la cantidad, el peso y el monto a pagar. El estado del encargo por defecto se guarda como “PENDIENTE”.
			
			El recepcionista debe imprimir el comprobante del registro del envío.
			
			Cuando la encomienda llega a su destino y se realiza el recojo, el recepcionista debe constatar que la persona que recoge sea la registrada en el sistema. Luego debe actualizar el estado del encargo a “ENTREGADO” y hacer entrega de la encomienda.
		} \\ \hline
		
		\textbf{Precondiciones:} El recepcionista debe acceder al sistema con su usuario y contraseña. & \textbf{Postcondiciones:} Ninguna. \\ \hline
		
		\multicolumn{2}{m{15cm}}{\textbf{Excepciones:} De no completar los datos en los registros, se mostrará un mensaje mencionando qué datos están vacíos y cuáles no han sido seleccionados.
			
		De no actualizar el estado del encargo entregado, éste se mantendrá como pendiente.
		} \\
	\end{longtable}
	\endgroup
	\vspace{-18pt}  % O el valor que necesites para ajustar
	% Nota personalizada fuera de `\caption*{}`
	\textbf{Nota}: Caso de uso que hace referencia al registro y gestión de encomiendas.
	
	\begingroup
	% \onehalfspacing
	
	\begin{longtable}{m{7.5cm}|m{7.5cm}}
		\caption[Especificación de casos de uso: Generar reporte]{\newline Especificación de casos de uso: Generar reporte} \label{tab:tabla3_10}\\
		\toprule
		\textbf{Caso de Uso:} Generar reporte & \textbf{Actores:} Administrador, personal de atención al cliente \\
		\midrule
		\endfirsthead
		\endhead		
		\bottomrule
		\endlastfoot
		
		\multicolumn{2}{m{15cm}}{\textbf{Descripción:} Este caso de uso hace referencia al proceso de generación de reportes necesarios para la gestión administrativa de la empresa.} \\ \hline
		
		\multicolumn{2}{m{15cm}}{\textbf{Secuencia Normal:}
			
			El usuario puede ver dentro del sistema sólo los reportes que su perfil le permita. Cabe mencionar que el perfil con todos los permisos u opciones es el del Administrador.
			
			El usuario selecciona el reporte que desea generar.
			
			El usuario ingresa los parámetros de búsqueda antes de generar el reporte.
			
			El sistema muestra los datos del reporte.
			
			El sistema imprime el reporte seleccionado.
		} \\ \hline
		
		\textbf{Precondiciones:} Los usuarios deben acceder al sistema con su usuario y contraseña. & \textbf{Postcondiciones:} Ninguna. \\ \hline
		
		\multicolumn{2}{m{15cm}}{\textbf{Excepciones:} Si el usuario y la contraseña no existen o si la contraseña no corresponde al usuario, se muestra una notificación de error solicitando nuevamente los datos.
		} \\
		
	\end{longtable}
	\endgroup
	\vspace{-18pt}  % O el valor que necesites para ajustar
	% Nota personalizada fuera de `\caption*{}`
	\textbf{Nota}: Caso de uso que describe el procedimiento para la generación de reportes.
		
	\subsection{Especificación de requerimientos}
	En esta etapa del proyecto, se presenta la especificación detallada de los requerimientos funcionales y no funcionales identificados para el sistema. Esta sección tiene como objetivo proporcionar una descripción completa de las capacidades y restricciones que deberá cumplir la solución, estableciendo así las bases para su posterior diseño e implementación.\\	
	\textbf{Requerimientos funcionales}
	\begingroup
	%\onehalfspacing	
	\begin{longtable}{m{1.5cm}|m{14cm}}
		\caption[Lista de Requerimientos Funcionales]{\newline Lista de Requerimientos Funcionales} \label{tab:tabla_requerimientos_funcionales}\\
		\toprule
		\textbf{ID} & \textbf{Requerimiento} \\
		\midrule
		\endfirsthead		
		\bottomrule
		\endlastfoot
		
		RF-01 & El sistema contará con un módulo de gestión de perfiles de usuarios, distinguiendo entre administradores y empleados. \\
		RF-02 & El sistema permitirá la gestión de rutas y horarios de los servicios de transporte. \\
		RF-03 & El sistema contará con un módulo de venta y reserva de pasajes para los usuarios. \\
		RF-04 & El sistema realizará la asignación de asientos en los vehículos de transporte. \\
		RF-05 & El sistema permitirá el registro de las encomiendas transportadas. \\
		RF-06 & El sistema gestionará la flota de buses, incluyendo información de los vehículos y conductores. \\		
	\end{longtable}	
	\endgroup
	\vspace{-18pt}  % O el valor que necesites para ajustar
	% Nota personalizada fuera de `\caption*{}`
	\textbf{Nota}: Esta tabla enumera los principales requerimientos funcionales del sistema propuesto, abarcando aspectos relacionados con usuarios, servicios de transporte, ventas y operaciones logísticas.\\		
	\textbf{Requerimientos no funcionales} \\	
	\begingroup
	%\onehalfspacing
	\begin{longtable}{m{1.4cm} m{14.5cm}}
		\caption[Lista de Requerimientos No Funcionales]{\newline Lista de Requerimientos No Funcionales} \label{tab:tabla_requerimientos_nofuncionales}\\
		\toprule
		\textbf{ID} & \textbf{Requerimiento} \\
		\midrule
		\endfirsthead		
		\bottomrule
		\endlastfoot
		
		RNF-01 & El sistema debe ser accesible desde múltiples navegadores (Chrome, Firefox) y celulares. \\
		RNF-02 & Capacidad de exportar reportes en formatos estándar (PDF, CSV, Excel). \\
		RNF-03 & Arquitectura de software modular y escalable para facilitar actualizaciones. \\
		
	\end{longtable}
	\endgroup
	
	\vspace{-18pt}
	\textbf{Nota}: Esta tabla enumera los principales requerimientos no funcionales del sistema.
		
	\subsection{Gestión de requerimientos}
	En esta etapa del proyecto, se presentan los diagramas de actividades que se utilizan para representar los flujos de trabajo relacionados con el manejo y control de los requerimientos. 
	
	%A continuación, en las figuras \ref{fig:DA_ingreso}, \ref{fig:DA_registro}, \ref{fig:DA_pasajes} y \ref{fig:DA_encomiendas}, se muestran los diagramas de actividades correspondientes.

	%\vspace{0.05cm} % Agregar 1 cm de espacio entre el párrafo y la figura
	
	\begin{figure}[!h] % 'H' del paquete 'float' para mantener posición	
		\caption[Diagrama de actividades - Ingresar al sistema]
		{\newline Diagrama de actividades del proceso de ingreso al sistema.} % Leyenda en la parte superior
		\centering
		\includegraphics[width=0.45\textwidth]{imagenes/cap_3/ingreso.drawio.png} % Inserta una imagen
		
		\begin{flushleft}
		\begin{doublespace}
			\hspace{1.20cm} \textbf{Nota.} El diagrama describe el flujo de actividades que realiza el usuario para iniciar sesión en el sistema. % Nota al pie para esta figura
		\end{doublespace}
		\end{flushleft}
		\vspace{-16pt}
		\label{fig:DA_ingreso} % Etiqueta para referencia cruzada
	\end{figure}

	\vspace{0.3cm} % Agregar 1 cm de espacio entre el párrafo y la figura
	
	\begin{figure}[!h] % 'H' del paquete 'float' para mantener posición	
		\caption[Diagrama de actividades - Venta y reserva de pasajes]
		{\newline Diagrama de actividades del proceso de venta y reserva de pasajes.} % Leyenda en la parte superior
		\centering
		\includegraphics[width=0.85\textwidth]{imagenes/cap_3/pasaje.drawio.png} % Inserta una imagen
		\begin{flushleft}
		\begin{doublespace}
			\hspace{1.20cm} \textbf{Nota.} El diagrama muestra el flujo de actividades desde el ingreeso al módulo de venta - reserva de pasajes hasta la confirmacion de la venta o reserva. % Nota al pie para esta figura
		\end{doublespace}
		\end{flushleft}
		\vspace{-16pt}
		\label{fig:DA_pasajes} % Etiqueta para referencia cruzada
	\end{figure}
	
	\vspace{0.3cm} % Agregar 1 cm de espacio entre el párrafo y la figura
	
	\begin{figure}[!h] % 'H' del paquete 'float' para mantener posición	
		\caption[Diagrama de actividades - Registro de encomiendas]
		{\newline Diagrama de actividades del proceso de registro del envío de encomiendas.} % Leyenda en la parte superior
		\centering
		\includegraphics[width=0.9\textwidth]{imagenes/cap_3/encomiendas.drawio.png} % Inserta una imagen
		
		\begin{flushleft}
		\begin{doublespace}
			\hspace{1.20cm} \textbf{Nota.} El diagrama representa las acciones necesarias para registrar una encomienda. % Nota al pie para esta figura
		\end{doublespace}
		\end{flushleft}
		\vspace{-16pt}
		\label{fig:DA_encomiendas} % Etiqueta para referencia cruzada
	\end{figure}
	
	\vspace{0.3cm} % Agregar 1 cm de espacio entre el párrafo y la figura
	
	\begin{figure}[!h] % 'H' del paquete 'float' para mantener posición	
		\caption[Diagrama de actividades - Registro de personal]
		{\newline Diagrama de actividades del proceso de registro de personal - clientes.} % Leyenda en la parte superior
		\centering
		\includegraphics[width=0.5\textwidth]{imagenes/cap_3/registro.drawio.png} % Inserta una imagen
		
		\begin{flushleft}
			\begin{doublespace}
			\hspace{1.20cm} \textbf{Nota.} El diagrama describe las actividades requeridas para registrar en el sistema a los usarios, tanto personal administrativo como clientes. % Nota al pie para esta figura
			\end{doublespace}
		\end{flushleft}
		\vspace{-40pt} % hace que se acerque mas el texto
		\label{fig:DA_registro} % Etiqueta para referencia cruzada
	\end{figure}
	
	En las figuras \ref{fig:DA_ingreso}, \ref{fig:DA_pasajes}, \ref{fig:DA_encomiendas}, y \ref{fig:DA_registro}, se muestran los diagramas de actividades con los flujos correspondientes a cada actividad.

\section{DISEÑO DEL SISTEMA Y DEL SOFTWARE}
	Para el modelado de datos ahora se procede a realizar el modelo entidad - relación que permite visualizar la estructura inicial de la base de datos.
	\subsection{Diagrama entidad - relación}
	A continuación en la figura \ref{fig:ent_atri} se muestran las entidades del sistema, cada entidad se representa con sus atributos específicos, se distinguen las claves primarias y los atributos descriptivos, estableciendo así la base para el almacenamiento de la información del sistema.
	
	\begin{figure}[!h] % 'H' del paquete 'float' para mantener posición	
		\caption[Diagrama Entidad-Relación ]
		{\newline Diagrama Entidad-Relación (Entidades y atributos)} % Leyenda en la parte superior
		\vspace{0.3cm}
		\centering
		\includegraphics[width=0.9\textwidth]{imagenes/cap_3/MER_ent_atri.png} % Inserta una imagen
		\vspace{0.6cm}
			\caption*{\textup{\textbf{Nota}: El diagrama muestra las entidades del sistema y sus atributos, sin incluir las relaciones entre ellas.}}
		\vspace{-0.6cm}
		\label{fig:ent_atri} % Etiqueta para referencia cruzada
	\end{figure}
	
	La figura \ref{fig:MER_com} representa el modelo entidad-relación, donde se visualizan las conexiones entre las diferentes entidades del sistema. Las líneas de relación muestran cómo las entidades interactúan entre sí, y la cardinalidad especificada en cada relación define las reglas de negocio y las restricciones del sistema.
	
	\begin{figure}[!h] % 'H' del paquete 'float' para mantener posición	
		\caption[Diagrama Entidad-Relación ]
		{\newline Diagrama Entidad-Relación} % Leyenda en la parte superior
		\vspace{0.3cm}
		\centering
		\includegraphics[width=1\textwidth]{imagenes/cap_3/MER_cali.drawio_completo.png} % Inserta una imagen
		\vspace{-0.1cm}
			\caption*{\textup{\textbf{Nota}: El diagrama presenta la estructura lógica de la base de datos del sistema.}}	
		\vspace{-0.6cm}
		\label{fig:MER_com} % Etiqueta para referencia cruzada
	\end{figure}
	
	\subsection{Diagrama relacional}
	En la figura \ref{fig:diag_rela} se presenta el diagrama relacional derivado del modelo entidad-relación mostrado en la figura \ref{fig:ent_atri} y figura \ref{fig:MER_com}.
	
	\vspace{0.3cm} % Agregar 1 cm de espacio entre el párrafo y la figura
	
	\begin{figure}[!h] % 'H' del paquete 'float' para mantener posición	
		\caption[Diagrama Relacional]
		{\newline Diagrama Relacional.} % Leyenda en la parte superior
		\centering
		\includegraphics[width=0.8\textwidth]{imagenes/cap_3/modelo_relacional.png} % Inserta una imagen
		
		\begin{flushleft}
			\hspace{1.20cm} \textbf{Nota.} El diagrama representa el modelo relacional derivado del diseño entidad-relacion. % Nota al pie para esta figura
		\end{flushleft}
		\vspace{-16pt}
		\label{fig:diag_rela} % Etiqueta para referencia cruzada
	\end{figure}
	
	Este diagrama refleja la estructura final de la base de datos, organizando las tablas y sus relaciones de acuerdo con las reglas de normalización. A través de este diagrama, se puede visualizar cómo se distribuyen los datos en las distintas tablas, asegurando la eficiencia en el almacenamiento y el manejo adecuado de las relaciones entre las entidades del sistema.
	
	\subsection{Diseño de la interfaz}
	
	El diseño de la interfaz se ha desarrollado para ofrecer una experiencia clara, intuitiva y accesible. Se ha tomado en cuenta la estructura lógica de los procesos del sistema, facilitando que el usuario recorra visualmente los elementos más importantes.
	
	Asimismo, la disposición de botones, formularios, menús y tablas responde a la lógica del escaneo visual en forma de "F", facilitando una rápida adaptación por parte de los usuarios. Este enfoque busca mejorar la eficiencia en la operación diaria, también reducir errores y aumentar la satisfacción del usuario con la herramienta.
	
	\vspace{0.3cm} % Agregar 1 cm de espacio entre el párrafo y la figura
	\begin{figure}[!h] % 'H' del paquete 'float' para mantener posición	
		\caption[Diseño de la página web principal del sistema]
		{\newline Diseño de la página web principal del sistema.} % Leyenda en la parte superior
		\centering
		\includegraphics[width=1\textwidth]{imagenes/cap_3/img_diseño/INTER00.png} % Inserta una imagen	
		\begin{flushleft}
			\hspace{1.20cm} \textbf{Nota.} Se presenta una estructura clara con accesos directos accesibles. % Nota al pie para esta figura
		\end{flushleft}
		\vspace{-24pt}
		\label{fig:inter00} % Etiqueta para referencia cruzada
	\end{figure}
	
	La página web principal, como se muestra en la figura \ref{fig:inter00}, presenta una estructura clara y visualmente atractiva para el usuario.
	
	\vspace{-1cm} % Agregar 1 cm de espacio entre el párrafo y la figura
	\begin{figure}[!h] % 'H' del paquete 'float' para mantener posición	
		\caption[Diseño del inicio de sesión]
		{\newline Diseño del inicio de sesión.} % Leyenda en la parte superior
		\centering
		\includegraphics[width=0.33\textwidth]{imagenes/cap_3/img_diseño/INTER01.png} % Inserta una imagen		
		\begin{flushleft}
			\hspace{1.20cm} \textbf{Nota.} Diseño preliminar del inicio de sesión. % Nota al pie para esta figura
		\end{flushleft}
		\vspace{-16pt}
		\label{fig:inter01} % Etiqueta para referencia cruzada
	\end{figure}
	\vspace{0.9cm} % Agregar 1 cm de espacio entre el párrafo y la figura
	
	La figura \ref{fig:inter01}, representa el diseño propuesto para la pantalla de inicio de sesión, enfocado en facilitar el ingreso de usuarios al sistema de forma segura.
	
	% \vspace{0.3cm} % Agregar 1 cm de espacio entre el párrafo y la figura
	\begin{figure}[!h] % 'H' del paquete 'float' para mantener posición	
		\caption[Diseño de la pantalla principal]
		{\newline Diseño de la pantalla principal} % Leyenda en la parte superior
		\centering
		\includegraphics[width=0.8\textwidth]{imagenes/cap_3/img_diseño/INTER02.png} % Inserta una imagen		
		\begin{flushleft}
			\hspace{1.20cm} \textbf{Nota.} Prototipo inicial de la pantalla principal con enfoque en la visualización de reportes. % Nota al pie para esta figura
		\end{flushleft}
		\vspace{-1pt}
		\label{fig:inter02} % Etiqueta para referencia cruzada
	\end{figure}
	\vspace{-0.6cm} % Agregar 1 cm de espacio entre el párrafo y la figura
	
	En la figura \ref{fig:inter02}, se observa el diseño de la vista principal, donde se plantean los elementos visuales para la consulta de reportes de ventas.
	
	
	\vspace{-1cm} % Agregar 1 cm de espacio entre el párrafo y la figura
	\begin{figure}[!h] % 'H' del paquete 'float' para mantener posición	
		\caption[Diseño interfaz Venta de pasajes]
		{\newline Diseño interfaz Venta de pasajes.} % Leyenda en la parte superior
		\centering
		\includegraphics[width=0.80\textwidth]{imagenes/cap_3/img_diseño/INTER03.png} % Inserta una imagen		
		\begin{flushleft}
			\hspace{1.20cm} \textbf{Nota.} Vista conceptual destinada a la venta de pasajes. % Nota al pie para esta figura
		\end{flushleft}
		\vspace{-16pt}
		\label{fig:inter03} % Etiqueta para referencia cruzada
	\end{figure}
	\vspace{0.9cm} % Agregar 1 cm de espacio entre el párrafo y la figura
	
	Como se muestra en la figura \ref{fig:inter03}, el diseño contempla una interfaz para la gestión de pasajes, organizada para agilizar el proceso de venta.
	
	% \vspace{0.3cm} % Agregar 1 cm de espacio entre el párrafo y la figura
	\begin{figure}[!h] % 'H' del paquete 'float' para mantener posición	
		\caption[Diseño interfaz Envío de encomiendas]
		{\newline Diseño interfaz Envío de encomiendas.} % Leyenda en la parte superior
		\centering
		\includegraphics[width=0.80\textwidth]{imagenes/cap_3/img_diseño/INTER04.png} % Inserta una imagen		
		\begin{flushleft}
			\hspace{1.20cm} \textbf{Nota.} Muestra la organización del formulario para gestionar las encomiendas. % Nota al pie para esta figura
		\end{flushleft}
		\vspace{-1pt}
		\label{fig:inter04} % Etiqueta para referencia cruzada
	\end{figure}
	\vspace{-0.6cm} % Agregar 1 cm de espacio entre el párrafo y la figura
	
	La figura \ref{fig:inter04}, corresponde al diseño preliminar de la pantalla destinada al registro y control de envíos de encomiendas.		
		
\section{IMPLEMENTACIÓN Y PRUEBA DE UNIDAD}

	El sistema ha sido desarrollado utilizando el lenguaje de programación Python junto con el framework Django, el cual se seleccionó por su arquitectura robusta basada en el patrón MVT (Model-View-Template). Esta estructura permitió una separación clara entre la lógica del negocio, la gestión de datos y la presentación visual, facilitando tanto el desarrollo modular como el mantenimiento del código. Además, Django proporciona herramientas integradas para la autenticación, la gestión de formularios, la protección contra ataques comunes como CSRF y XSS, y un panel de administración muy útil para la gestión de datos durante el desarrollo.
	
	Durante el proceso de implementación se aplicaron pruebas de unidad a nivel de modelo y vista, con el objetivo de validar que cada componente funcione de forma independiente y según lo esperado. Estas pruebas permitieron detectar y corregir errores en etapas tempranas del desarrollo, mejorando la calidad general del sistema. Se utilizaron las herramientas de prueba integradas en Django, como el módulo TestCase, lo que aseguró una cobertura adecuada del código y una base sólida para futuras extensiones del sistema.
	
	\subsection{Gestión de usuarios}
	La gestión de usuarios se implementa mediante el Programa 3.1, el cual define la vista encargada de controlar el acceso y manejo de las cuentas de usuario dentro del sistema. Esta vista permite realizar operaciones como el inicio de sesión, el cierre de sesión, el registro de nuevos usuarios y la edición de perfiles, todo ello con validaciones que aseguran la integridad de los datos ingresados. El programa se comunica directamente con el modelo de usuarios y aprovecha las herramientas integradas del framework para simplificar el manejo de credenciales y sesiones activas.
	
	Además, el sistema establece distintos niveles de permisos y roles que permiten controlar qué acciones puede realizar cada tipo de usuario. Por ejemplo, ciertos módulos solo están disponibles para administradores, mientras que otros pueden ser accesibles por operadores o personal autorizado. Esto refuerza la seguridad y la organización interna del sistema, permitiendo una administración más eficiente. El diseño modular de este componente facilita futuras ampliaciones, como la integración con sistemas de autenticación externos o la implementación de medidas adicionales como verificación en dos pasos.
		
	\textbf{Programa 3.1}
	
	\textit{Creacion de Usuarios.} % Título y subtítulo alineados
	\vspace{0.3cm} % Espaciado opcional entre el título y el código
	\begin{lstlisting}[lineskip=-1pt]
		class UserCreateView(LoginRequiredMixin, ValidatePermissionRequiredMixin, CreateView):
			model = User
			form_class = UserForm
			template_name = 'user/create.html'
			success_url = reverse_lazy('user:user_list')
			permission_required = 'user.add_user'
			url_redirect = success_url
			
			@method_decorator(csrf_exempt)
			def dispatch(self, request, *args, **kwargs):
				return super().dispatch(request, *args, **kwargs)
			
			def post(self, request, *args, **kwargs):
				data = {}
				try:
					action = request.POST['action']
					if action == 'add':
						form = self.get_form()
						data = form.save()
					else:
						data['error'] = 'No ha ingresado a ninguna opción'
				except Exception as e:
					data['error'] = str(e)
				return JsonResponse(data)
			
			def get_context_data(self, **kwargs):
				context = super().get_context_data(**kwargs)
				context['title'] = 'Creación de un Usuario'
				context['entity'] = 'Usuarios'
				context['list_url'] = self.success_url
				context['action'] = 'add'
				return context
	\end{lstlisting}
	
	\textbf{Nota:} Código que permite la creación de nuevos usuarios al sistema.
	
	\subsection{Gestión de rutas}
	
		El sistema de rutas se implementa a través del Programa 3.2, que establece los modelos necesarios para almacenar la información de rutas correspondientes, y el Programa 3.3, que proporciona las vistas para la gestión de estas entidades. La combinación de estos programas permite una administración eficiente de las rutas y sus programaciones.
		
		\textbf{Programa 3.2}
		
		\textit{Modelo gestión de rutas.} % Título y subtítulo alineados
		\vspace{0.3cm} % Espaciado opcional entre el título y el código
		\begin{lstlisting}[lineskip=-1pt]
			
			class Route(BaseModel):
				origin = models.CharField(
				max_length=100, verbose_name='Origen')
				destination = models.CharField(
				max_length=500, null=True, blank=True, verbose_name='Destino')
				estimated_time = models.CharField(
				max_length=100, null=True, blank=True, verbose_name='Tiempo estimado')
				
				def __str__(self):
					return self.origin
				
				def save(self, force_insert=False, force_update=False, using=None,
					update_fields=None):
					user = get_current_user()
					if user is not None:
						if not self.pk:
							self.user_creation = user
						else:
							self.user_updated = user
					super(Route, self).save()
				
				def toJSON(self):
					item = model_to_dict(self, exclude=['user_creation', 'user_updated'])
					return item
				
				class Meta:
					db_table = 'Rutas'
					verbose_name = 'Ruta'
					verbose_name_plural = 'Rutas'
					ordering = ['id']
		\end{lstlisting}
		
		\textbf{Nota:} Definición del modelo de datos que representa las rutas en el sistema.
		
		\vspace{1cm}
		
		\textbf{Programa 3.3}
		
		\textit{Vista de la gestión de rutas.} % Título y subtítulo alineados
		\vspace{0.3cm} % Espaciado opcional entre el título y el código
		\begin{lstlisting}[lineskip=-1pt]
			
			class RouteListView(LoginRequiredMixin, ValidatePermissionRequiredMixin, ListView):
				model = Route
				template_name = 'route/list.html'
				permission_required = 'calibus.view_route'
				
			@method_decorator(csrf_exempt)
			def dispatch(self, request, *args, **kwargs):
				return super().dispatch(request, *args, **kwargs)
			
			def post(self, request, *args, **kwargs):
				data = {}
				try:
					action = request.POST['action']
					if action == 'searchdata':
						data = []
						for i in Route.objects.all():
						data.append(i.toJSON())
					else:
						data['error'] = 'Ha ocurrido un error'
				except Exception as e:
					data['error'] = str(e)
				return JsonResponse(data, safe=False)
			
			def get_context_data(self, **kwargs):
				context = super().get_context_data(**kwargs)
				context['title'] = 'Listado de rutas'
				context['create_url'] = reverse_lazy('calibus:route_create')
				context['list_url'] = reverse_lazy('calibus:route_list')
				context['entity'] = 'Rutas'
				return context
		\end{lstlisting}
		
		\textbf{Nota:} Código de la interfaz para listar rutas disponibles en el sistema.
	
	\subsection{Gestión de buses}
	
	La gestión de la flota de buses se implementa a través del Programa 3.4, que define el modelo para los buses y su información asociada, y el Programa 3.5, que implementa la vista para la creación de los buses. Estos programas permiten un control completo sobre la flota de buses. Se pueden registrar y editar datos como la placa, capacidad, estado operativo, etc. Además, esta gestión centralizada facilita el seguimiento del mantenimiento, la planificación de horarios y la disponibilidad de unidades, contribuyendo a una operación más eficiente y organizada.
	
	\textbf{Programa 3.4}
	
	\textit{Modelo para la creación de buses.} % Título y subtítulo alineados
	\vspace{0.3cm} % Espaciado opcional entre el título y el código
		\begin{lstlisting}[lineskip=-1pt]
			class Bus(models.Model):
				license_plate = models.CharField(max_length=10, verbose_name="Placa")
				chassis_number = models.CharField(max_length=20, verbose_name="Número de chasis")
				engine_number = models.CharField(max_length=20, verbose_name="Número de motor")
				model = models.CharField(max_length=30, verbose_name="Modelo")
				color = models.CharField(max_length=20, verbose_name="Color")
				brand = models.CharField(max_length=30, verbose_name="Marca")
				capacity = models.PositiveIntegerField(verbose_name="Capacidad")
				year = models.PositiveIntegerField(verbose_name="Año")
				status = models.BooleanField(default=True, verbose_name="Estado")
			
			def __str__(self):
				return self.license_plate
			
			def toJSON(self):
				item = model_to_dict(self)
				return item
			
			class Meta:
				db_table = "Buses"
				verbose_name = "Bus"
				verbose_name_plural = "Buses"
				ordering = ["id"]
		\end{lstlisting}
	
	\textbf{Nota:} Código del modelo de buses de la base de datos.
	
	\textbf{Programa 3.5}
	
	\textit{Vista de la creación de buses.} % Título y subtítulo alineados
	\vspace{0.3cm} % Espaciado opcional entre el título y el código
	\begin{lstlisting}[lineskip=-1pt]
		class BusCreateView(LoginRequiredMixin, ValidatePermissionRequiredMixin, CreateView):
			model = Bus
			form_class = BusForm
			template_name = 'bus/create.html'
			success_url = reverse_lazy('calibus:bus_list')
			permission_required = 'calibus.add_bus'
			url_redirect = success_url
		
		def dispatch(self, request, *args, **kwargs):
			return super().dispatch(request, *args, **kwargs)
		
		def post(self, request, *args, **kwargs):
			data = {}
			try:
				action = request.POST['action']
				if action == 'add':
					form = self.get_form()
					data = form.save()
				else:
					data['error'] = 'No ha ingresado a ninguna opción'
			except Exception as e:
				data['error'] = str(e)
			return JsonResponse(data)
		
		def get_context_data(self, **kwargs):
			context = super().get_context_data(**kwargs)
			context['title'] = 'Creación de un Bus'
			context['entity'] = 'Buses'
			context['list_url'] = self.success_url
			context['action'] = 'add'
			return context
	\end{lstlisting}
	
	\textbf{Nota:} Código de la vista para la creación de buses.
	
	\subsection{Gestión de viajes}
	
	\textbf{Programa 3.6}
	
	\textit{Modelo para la gestión de viajes.} % Título y subtítulo alineados
	\vspace{0.3cm} % Espaciado opcional entre el título y el código
	\begin{lstlisting}[lineskip=-1pt]
		class Travel(models.Model):
			routeID = models.ForeignKey(Route, on_delete=models.CASCADE, verbose_name="Ruta")
			busID = models.ForeignKey(Bus, on_delete=models.CASCADE, verbose_name="Bus")
			departure = models.DateField(default=datetime.now, verbose_name="Fecha de salida")
			departure_time = models.TimeField(default=current_time, verbose_name="Hora de salida")
			arrival = models.DateField(default=datetime.now, verbose_name="Fecha de llegada")
			status = models.CharField(max_length=10,choices=travel_status_choices,default="active",verbose_name="Estado")		
		def __str__(self):
			return f"({self.departure} - {self.departure_time}) {self.busID.license_plate} -> {self.routeID.destination}"		
		def toJSON(self):
			item = model_to_dict(self)
			item["route"] = self.routeID.toJSON()
			return item		
		class Meta:
			db_table = "Viajes"
			verbose_name = "Viaje"
			verbose_name_plural = "Viajes"
			ordering = ["id"]
	\end{lstlisting}
	
	\textbf{Nota:} Modelo para pa gestión de viajes.
	
	La gestión de viajes se implementa mediante el Programa 3.6, que define el modelo con la información relevante de cada viaje, como fecha, hora, ruta y bus asignado. El Programa 3.7 complementa esta funcionalidad al presentar una vista tipo lista donde se muestran todos los viajes registrados. Esta vista permite filtrar, editar o eliminar viajes de forma rápida y ordenada.
	
	\textbf{Programa 3.7}
	
	\textit{Vista para el listado de viajes.} % Título y subtítulo alineados
	\vspace{0.3cm} % Espaciado opcional entre el título y el código
	\begin{lstlisting}[lineskip=-1pt]
		class TravelListView(LoginRequiredMixin, ValidatePermissionRequiredMixin, ListView):
			model = Travel
			template_name = 'travel/list.html'
			permission_required = 'calibus.view_travel'
		
		@method_decorator(csrf_exempt)
		def dispatch(self, request, *args, **kwargs):
			return super().dispatch(request, *args, **kwargs)
		
		def post(self, request, *args, **kwargs):
			data = {}
			try:
				action = request.POST['action']
				if action == 'searchdata':
					data = []
					for i in Travel.objects.all():
						data.append(i.toJSON())
				else:
				data['error'] = 'Ha ocurrido un error'
			except Exception as e:
				data['error'] = str(e)
			return JsonResponse(data, safe=False)
		
		def get_context_data(self, **kwargs):
			context = super().get_context_data(**kwargs)
			context['title'] = 'Listado de Viajes'
			context['create_url'] = reverse_lazy('calibus:travel_create')
			context['list_url'] = reverse_lazy('calibus:travel_list')
			context['entity'] = 'Viajes'
			context['parent'] = 'empresa'
			context['segment'] = 'viaje'
			return context
	\end{lstlisting}
	
	\textbf{Nota:} Vista tipo lista que permita visualizar, filtrar y administrar los viajes registrados en el sistema.
	
	\subsection{Venta y reserva de pasajes}
		
		El sistema de venta y reservas se implementa mediante el Programa 3.8, que define el modelo de clientes para los datos de la venta y/o reserva de pasajes, el Programa 3.9, muestra el modelo para la gestión de viajes. Estos programas trabajan en conjunto para garantizar la integridad de las transacciones y prevenir conflictos en ventas y/o reservas simultáneas.
		
		\textbf{Programa 3.8}
		
		\textit{Modelo para la gestión de pasajes.} % Título y subtítulo alineados
		\vspace{0.3cm} % Espaciado opcional entre el título y el código
		\begin{lstlisting}[lineskip=-1pt]
			
			class Ticket(models.Model):
				clientID = models.ForeignKey(Client, on_delete=models.CASCADE)
				travelID = models.ForeignKey(Travel, on_delete=models.CASCADE)
				purchase_date = models.DateField(default=datetime.now, verbose_name="Fecha de compra")
				purchase_time = models.TimeField(default=current_time, verbose_name="Hora de compra")
				ticket_type = models.CharField(max_length=20,choices=ticket_type_choices,default="libre",verbose_name="Tipo de pasaje",)
				total_price = models.DecimalField(default=0.00, max_digits=9, decimal_places=2,verbose_name="Precio total")
				status = models.BooleanField(default=True, verbose_name="Estado")			
				def __str__(self):
					return f"Ticket de {self.clientID.names} para el viaje {self.travelID.id}"
			
				def toJSON(self):
					data = model_to_dict(self)
					data["clientID"] = (f"{self.clientID.names} {self.clientID.surnames}"  # Combina nombres y apellidos)
					data["travelID"] = (f"{self.travelID.routeID.origin} -> {self.travelID.routeID.destination} ({self.travelID.departure}{self.travelID.departure_time})")
					return data
				class Meta:
					db_table = "Pasajes"
					verbose_name = "Pasaje"
					verbose_name_plural = "Pasajes"
					ordering = ["id"]			
		\end{lstlisting}
		
		\textbf{Nota:} Modelo que gestiona la venta y reserva de pasajes, incluyendo datos del pasajero, asiento, viaje.
		
		\textbf{Programa 3.9}
		
		\textit{Vista para el registro de pasajes.} % Título y subtítulo alineados
		\vspace{0.3cm} % Espaciado opcional entre el título y el código
		\begin{lstlisting}[lineskip=-1pt]
			class TicketCreateView(LoginRequiredMixin, ValidatePermissionRequiredMixin, CreateView):
				model = Ticket
				form_class = TicketForm
				template_name = "ticket/create.html"
				success_url = reverse_lazy("index")
				permission_required = "calibus.add_ticket"
				url_redirect = success_url			
			def dispatch(self, request, *args, **kwargs):
				return super().dispatch(request, *args, **kwargs)			
			def post(self, request, *args, **kwargs):
				data = {}
				try:
					action = request.POST["action"]
					if action == "add":
						form = self.get_form()
						data = form.save()
					else:
						data["error"] = "No ha ingresado a ninguna opción"
				except Exception as e:
					data["error"] = str(e)
				return JsonResponse(data)			
			def get_context_data(self, **kwargs):
				context = super().get_context_data(**kwargs)
				context["title"] = "Registro de Pasajes"
				context["entity"] = "Pasajes"
				context["list_url"] = self.success_url
				context["action"] = "add"
				travel_id = self.request.GET.get("travel")
				if travel_id:
					try:
						travel = Travel.objects.get(pk=travel_id)
						bus = travel.busID
						context["travel"] = travel
						context["bus"] = bus
						context["total_seats"] = bus.capacity
						print("DEBUG bus.capacity:", bus.capacity)  
					except Travel.DoesNotExist:
						context["travel"] = None
						context["bus"] = None
						context["total_seats"] = 0
				else:
					context["travel"] = None
					context["bus"] = None
					context["total_seats"] = 0
				return context
		\end{lstlisting}
		
		\textbf{Nota:} Vista para registrar ventas o reservas de pasajes, con validaciones para evitar duplicidades y conflictos de asignación.
			
	\subsection{Registro de encomiendas}
	
		El sistema de encomiendas se implementa mediante el Programa 3.10, que define el modelo para el registro y seguimiento de encomiendas, y el Programa 3.11, que proporciona la vista para la creación y gestión de encomiendas. Estos programas permiten un seguimiento detallado de cada encomienda en el sistema.
		
		\textbf{Programa 3.10}
		
		\textit{Modelo para el registro de encomiendas.} % Título y subtítulo alineados
		\vspace{0.3cm} % Espaciado opcional entre el título y el código
		\begin{lstlisting}[lineskip=-1pt]
			class Parcel(models.Model):
				senderID = models.ForeignKey(Client,on_delete=models.CASCADE,related_name="sent_parcels")
				receiverID = models.ForeignKey(Client, on_delete=models.CASCADE,related_name="received_parcels")
				travelID = models.ForeignKey(Travel, on_delete=models.CASCADE)
				date_joined = models.DateField(default=datetime.now, verbose_name="Fecha de registro")
				status = models.CharField(max_length=30,choices=parcel_choices,default="pending",verbose_name="Estado",blank=True,)
				total = models.DecimalField(default=0.00,max_digits=10,decimal_places=2,verbose_name="Total Precio de Envío",)
			
				def __str__(self):
					return self.senderID.names
						
				def toJSON(self):
					data = model_to_dict(self)
					data["senderID"] = (f"{self.senderID.names} {self.senderID.surnames}"  # Combina nombres y apellidos)
					data["receiverID"] = f"{self.receiverID.names} {self.receiverID.surnames}"
					data["travelID"] = (f"{self.travelID.routeID.origin} -> {self.travelID.routeID.destination} ({self.travelID.departure} {self.travelID.departure_time})")
					return data
			
				class Meta:
					db_table = "Encomiendas"
					verbose_name = "Encomienda"
					verbose_name_plural = "Encomiendas"
					ordering = ["id"]
		\end{lstlisting}
		
		\textbf{Nota:} Modelo que estructura los datos de las encomiendas, incluyendo remitente, consignatario y datos del envío.
		
		\textbf{Programa 3.11}
		
		\textit{Vista para la creación de encomiendas} % Título y subtítulo alineados
		\vspace{0.3cm} % Espaciado opcional entre el título y el código
		\begin{lstlisting}[lineskip=-1pt]
		class ParcelCreateView(LoginRequiredMixin, ValidatePermissionRequiredMixin, CreateView):
			model = Parcel
			form_class = ParcelForm
			template_name = 'parcel/create.html'
			success_url = reverse_lazy('index')
			permission_required = 'calibus.add_parcel'
			url_redirect = success_url
			
			def dispatch(self, request, *args, **kwargs):
			return super().dispatch(request, *args, **kwargs)
			
			def post(self, request, *args, **kwargs):
				data = {}
				try:
					action = request.POST['action']
					if action == 'add':
						form = self.get_form()
						data = form.save()
					else:
						data['error'] = 'No ha ingresado a ninguna opción'
				except Exception as e:
					data['error'] = str(e)
				return JsonResponse(data)
			
			def get_context_data(self, **kwargs):
				context = super().get_context_data(**kwargs)
				context['title'] = 'Creación una Encomienda'
				context['entity'] = 'Encomiendas'
				context['list_url'] = self.success_url
				context['action'] = 'add'
				return context
		\end{lstlisting}
		
		\textbf{Nota:} Vista que permite registrar, actualizar y gestionar el seguimiento de encomiendas dentro del sistema.
	
	\subsection{Gestión de clientes}
	
		La gestión de clientes se implementa mediante el Programa 3.12, que define el modelo encargado de almacenar la información personal y de contacto de los clientes.  Complementando esta funcionalidad, el Programa 3.13 proporciona la vista para el registro y edición de clientes, permitiendo una administración eficiente y centralizada de los datos esenciales para los procesos de envíos de encomiendas.
	
	\textbf{Programa 3.12}
	
	\textit{Modelo para la gestión de clientes.} % Título y subtítulo alineados
	\vspace{0.3cm} % Espaciado opcional entre el título y el código
	\begin{lstlisting}[lineskip=-1pt]
		class Client(models.Model):
			names = models.CharField(max_length=150, verbose_name="Nombres")
			surnames = models.CharField(max_length=150, verbose_name="Apellidos")
			ci = models.CharField(
			max_length=10, unique=True, verbose_name="Cédula de Identidad")
			nationality = models.CharField(max_length=20, default="Desconocido",verbose_name="Nacionalidad")
			date_of_birth = models.DateField(default=datetime.now, verbose_name="Fecha de nacimiento")
			phone = models.CharField(max_length=10, verbose_name="Teléfono")
			email = models.CharField(max_length=100, verbose_name="Correo electrónico")
			gender = models.CharField(max_length=10, choices=gender_choices, default="male",verbose_name="Sexo")
		
			def __str__(self):
				return self.names
			
			def toJSON(self):
				item = model_to_dict(self)
				item["gender"] = {"id": self.gender, "name": self.get_gender_display()}
				return item
		
			class Meta:
				db_table = "Clientes"
				verbose_name = "Cliente"
				verbose_name_plural = "Clientes"
				ordering = ["id"]
	\end{lstlisting}
	
	\textbf{Nota:} Modelo que almacena los datos personales de los clientes.
	
	\textbf{Programa 3.13}
	
	\textit{Vista para el registro de clientes.} % Título y subtítulo alineados
	\vspace{0.3cm} % Espaciado opcional entre el título y el código
	\begin{lstlisting}[lineskip=-1pt]
		class ClientCreateView(LoginRequiredMixin, ValidatePermissionRequiredMixin, CreateView):
			model = Client
			form_class = ClientForm
			template_name = 'client/create.html'
			success_url = reverse_lazy('calibus:client_list')
			permission_required = 'calibus.add_client'
			url_redirect = success_url
		
			def dispatch(self, request, *args, **kwargs):
				return super().dispatch(request, *args, **kwargs)
		
			def post(self, request, *args, **kwargs):
				data = {}
				try:
					action = request.POST['action']
					if action == 'add':
						form = self.get_form()
						data = form.save()
					else:
						data['error'] = 'No ha ingresado a ninguna opción'
				except Exception as e:
					data['error'] = str(e)
				return JsonResponse(data)
			
			def get_context_data(self, **kwargs):
				context = super().get_context_data(**kwargs)
				context['title'] = 'Creación un Cliente'
				context['entity'] = 'Clientes'
				context['list_url'] = self.success_url
				context['action'] = 'add'
				return context
	\end{lstlisting}
	
	\textbf{Nota:} Vista destinada al registro y edición de cleintes, facilitando su administración dentro del sistema.	
	
\section{PRUEBA DE SISTEMA}
	\subsection{Pruebas unitarias}
	
	En el marco del desarrollo del proyecto, enfocado en la gestión de usuarios, la venta de pasajes y el envío de encomiendas, las pruebas unitarias desempeñan un papel fundamental para garantizar la calidad y fiabilidad del sistema. Implementadas mediante la herramienta unittest de Python, estas pruebas aseguran que cada componente funcione correctamente y cumpla con los requisitos establecidos.
	
	Las pruebas se aplicaron principalmente a los modelos y vistas, evaluando el comportamiento de funciones clave como la autenticación de usuarios, la venta o reserva de pasajes y el registro de encomiendas. Este proceso no solo valida que los datos se almacenen y gestionen adecuadamente, sino que también fortalece la estabilidad del sistema ante posibles modificaciones futuras.
	
	El Programa 3.14 contiene las pruebas relacionadas con la gestión de usuarios. Este archivo incluye los siguientes casos de prueba:
	
	\textbf{Programa 3.14}
	
	\textit{Pruebas de Gestión de Usuarios.} % Título y subtítulo alineados
	\vspace{0.3cm} % Espaciado opcional entre el título y el código
	\begin{lstlisting}[lineskip=-1pt]
		class PruebasGestionUsuario(unittest.TestCase):
		def setUp(self):
			# Inicializa el servicio de usuario que vamos a probar
			self.servicio_usuario = ServicioUsuario()
		
		# Prueba 1: Verificar que un usuario se puede registrar correctamente
		def test_registro_usuario_exitoso(self):
		# Datos de prueba para el registro
		datos_usuario = {
			"email": "bladi.ram@gmail.com",
			"password": "Contraseña123",
			"nombre": "Bladimir",
			"apellido": "Ramos",
			"telefono": "1234567"
		}
		# Intenta registrar al usuario
		resultado = self.servicio_usuario.registrar_usuario(datos_usuario)
		
		# Verifica que el registro fue exitoso
		self.assertTrue(resultado.exito)  
		self.assertIsNotNone(resultado.id_usuario)
		
		# Prueba 2: Verificar que un usuario puede iniciar sesión correctamente
		def test_login_usuario_exitoso(self):
		# Credenciales de prueba
		credenciales = {
			"email": "bladi.ram@gmail.com",
			"password": "Contraseña123"
		}
		# Intenta iniciar sesión
		resultado = self.servicio_usuario.iniciar_sesion(credenciales)
		
		# Verifica que el login fue exitoso
		self.assertTrue(resultado.exito)  
		self.assertIsNotNone(resultado.token_sesion)  
		
		# Prueba 3: Verificar que el sistema rechaza credenciales incorrectas
		def test_credenciales_invalidas(self):
		# Credenciales incorrectas de prueba
		credenciales = {
			"email": "prueba@gmail.com",
			"password": "ContraseñaIncorrecta"
		}
		# Intenta iniciar sesión con credenciales incorrectas
		resultado = self.servicio_usuario.iniciar_sesion(credenciales)
		
		# Verifica que el login falló como se esperaba
		self.assertFalse(resultado.exito)  
		self.assertEqual(resultado.mensaje_error, "Credenciales inválidas")  
	\end{lstlisting}
	
	\textbf{Nota:} Pruebas unitarias para validar el funcionamiento del registro de usuarios.
	
	El Programa 3.15 evalúa funcionalidades asociadas a la venta y reserva de pasajes. 		
	
	\textbf{Programa 3.15}
	
	\textit{Pruebas de venta y reserva de pasajes.} % Título y subtítulo alineados
	\vspace{0.3cm} % Espaciado opcional entre el título y el código
	\begin{lstlisting}[lineskip=-1pt]
		class PruebasReservaPasajes(unittest.TestCase):
		
		def setUp(self):
		self.servicio_reservas = ServicioReservas()
		self.id_usuario = "12"
		
		def test_verificar_disponibilidad_pasajes(self):
		parametros_busqueda = {
			"origen": "La Paz",
			"destino": "Arica",
			"fecha": datetime(2024, 12, 1),
			"pasajes": 2
		}
		resultado = self.servicio_reservas.verificar_disponibilidad(parametros_busqueda)
		self.assertTrue(resultado.exito)
		self.assertGreater(len(resultado.pasajes_disponibles), 0)
		
		def test_reserva_pasaje_exitosa(self):
		datos_reserva = {
			"id_usuario": self.id_usuario,
			"id_viaje": "13",
			"pasajeros": [
			{"name": "Juan Pérez", "asiento": "12A"},
			{"name": "María Pérez", "asiento": "12B"}
			],
			"info_pago": {
				"method": "Efectivo",
				"mount": 300.00
			}
		}
		resultado = self.servicio_reservas.reservar_pasaje(datos_reserva)
		self.assertTrue(resultado.exito)
		self.assertIsNotNone(resultado.id_reserva)
		self.assertIsNotNone(resultado.codigo_confirmacion)
		
		def test_cancelacion_reserva(self):
		id_reserva = "23"
		resultado = self.servicio_reservas.cancelar_reserva(id_reserva, self.id_usuario)
		self.assertTrue(resultado.exito)
		self.assertIsNotNone(resultado.monto_reembolso)	
	\end{lstlisting}
	
	\textbf{Nota:} Pruebas que verifican el proceso de venta y reserva de pasajes, incluyendo validaciones de disponibilidad y asignación de asientos.
	
	\vspace{2cm}
	
	El Programa 3.16 está dedicado a las pruebas del módulo de envío de encomiendas.
	
	\textbf{Programa 3.16}
	
	\textit{Pruebas de Gestión de Encomiendas.} % Título y subtítulo alineados
	\vspace{0.3cm} % Espaciado opcional entre el título y el código
	\begin{lstlisting}[lineskip=-1pt]
		
		class PruebaModeloParcel(TestCase):
		
			def setUp(self):
			
				self.remitente = Client.objects.create(
				names="Carlos", surnames="Ramírez", email="carlos@example.com"
				)
				self.destinatario = Client.objects.create(
				names="Lucía", surnames="Gómez", email="lucia@example.com"
				)
				self.viaje = Travel.objects.create(
				routeID_id=1,
				departure=datetime(2024, 12, 10),
				departure_time="10:00"
				)
				
				self.encomienda = Parcel.objects.create(
				senderID=self.remitente,
				receiverID=self.destinatario,
				travelID=self.viaje,
				status="pending",
				total=120.50
				)
			
			def test_creacion_encomienda(self):
			
				"""Verifica que se crea correctamente una encomienda"""
				self.assertEqual(self.encomienda.senderID.names, "Carlos")
				self.assertEqual(self.encomienda.receiverID.names, "Lucía")
				self.assertEqual(str(self.encomienda), "Carlos")
				self.assertEqual(self.encomienda.status, "pending")
				self.assertGreater(self.encomienda.total, 0)
			
			def test_representacion_json(self):
			
				"""Verifica el método toJSON() de encomienda"""
				json_data = self.encomienda.toJSON()
				self.assertIn("senderID", json_data)
				self.assertIn("receiverID", json_data)
				self.assertIn("travelID", json_data)
				self.assertEqual(json_data["senderID"], "Carlos Ramírez")
				
	\end{lstlisting}
	
	\textbf{Nota:} Pruebas unitarias orientadas a verificar el registro correcto de las encomiendas.
		
	\subsection{Resultados de las pruebas}
	
	En la figura \ref{fig:prue_uni}, se observa los resultados de las tres pruebas realizadas.
	
	\vspace{0.2cm} % Agregar 1 cm de espacio entre el párrafo y la figura
	
	\begin{figure}[!h] % 'H' del paquete 'float' para mantener posición	
		\caption[Pruebas Unitarias]
		{\newline Pruebas Unitarias.} % Leyenda en la parte superior
		\centering
		\includegraphics[width=0.65\textwidth]{imagenes/cap_3/pruebas_unitarias.png} % Inserta una imagen
		\begin{flushleft}
			\hspace{1.20cm} \textbf{Nota.} Resultados generados tras la ejecución de las pruebas unitarias. % Nota al pie para esta figura
		\end{flushleft}
		%\vspace{-16pt}
		\label{fig:prue_uni} % Etiqueta para referencia cruzada
	\end{figure}
	\vspace{-20pt} % Agregar 1 cm de espacio entre el párrafo y la figura
	
	Estas pruebas no solo facilitaron la detección de errores en etapas tempranas del desarrollo, sino que también incrementan la confianza en el sistema, garantizando que cada módulo funcione de manera independiente y que los cambios no introduzcan fallos. Además, al integrarlas en un flujo de trabajo continuo, permiten un desarrollo seguro, reduciendo los tiempos de corrección.
	
 \section{OPERACIÓN Y MANTENIMIENTO}
 
 	La fase de Operación y Mantenimiento corresponde a la etapa final del modelo en cascada, en la cual el sistema entra en funcionamiento real y es utilizado por los usuarios finales, durante esta fase, se supervisa el comportamiento del sistema para garantizar su correcto desempeño en un entorno de producción. Además, se da seguimiento a posibles incidencias, errores o mejoras solicitadas por los usuarios, lo que permite realizar ajustes que aseguren la estabilidad y continuidad operativa del sistema.
 	
 	En esta etapa, se han documentado las interfaces implementadas mediante capturas de pantalla representativas, como se muestra en las figuras correspondientes, la recopilación de estas vistas proporciona evidencia visual del sistema en operación, permitiendo evaluar su funcionalidad desde el punto de vista práctico, así como servir de base para futuras tareas de mantenimiento y mejora continua.
 
 % \vspace{0.3cm} % Agregar 1 cm de espacio entre el párrafo y la figura
 \begin{figure}[!h] % 'H' del paquete 'float' para mantener posición	
 	\caption[Página web - Cali Internacional]
 	{\newline Página web - Cali Internacional.} % Leyenda en la parte superior
 	\centering
 	\includegraphics[width=0.9\textwidth]{imagenes/cap_3/Img_calibus/CALIBUS00.png} % Inserta una imagen
 	\begin{flushleft}
 		\begin{doublespace}
 			\hspace{1.20cm} \textbf{Nota.} Página principal de la web de Cali Internacional. % Nota al pie para esta figura
 		\end{doublespace}
 	\end{flushleft}
 	\vspace{-40pt} % hace que se acerque mas el texto
 	\label{fig:cali00} % Etiqueta para referencia cruzada
 \end{figure}
 
 %\vspace{-0.6cm} % Agregar 1 cm de espacio entre el párrafo y la figura
 
 En la figura \ref{fig:cali00} se ve la página web inicial de la Empresa de transportes Cali Internacional, posteriormente en la Figura \ref{fig:cali01} se observa la página de inicio de sesión.
 
 \vspace{0.3cm} % Agregar 1 cm de espacio entre el párrafo y la figura
 
 \begin{figure}[!h] % 'H' del paquete 'float' para mantener posición	
 	\caption[Inicio de sesión]
 	{\newline Inicio de sesión.} % Leyenda en la parte superior
 	\centering
 	\includegraphics[width=0.75\textwidth]{imagenes/cap_3/Img_calibus/CALIBUS01.png} % Inserta una imagen
 	
 	\begin{flushleft}
 		\begin{doublespace}
 			\hspace{1.20cm} \textbf{Nota.} Interfaz de inicio de sesión para el acceso seguro al sistema. % Nota al pie para esta figura
 		\end{doublespace}
 	\end{flushleft}
 	\vspace{-40pt} % hace que se acerque mas el texto
 	\label{fig:cali01} % Etiqueta para referencia cruzada
 \end{figure}
 
 
 \begin{figure}[!h] % 'H' del paquete 'float' para mantener posición	
 	\caption[Panel principal del sistema]
 	{\newline Panel principal del sistema.} % Leyenda en la parte superior
 	\centering
 	\includegraphics[width=0.75\textwidth]{imagenes/cap_3/Img_calibus/CALIBUS02.png} % Inserta una imagen
 	
 	\begin{flushleft}
 		\hspace{1.20cm} \textbf{Nota.} Panel principal que resume la información operativa y financiera. % Nota al pie para esta figura
 	\end{flushleft}
 	\vspace{-16pt}
 	\label{fig:cali02} % Etiqueta para referencia cruzada
 \end{figure}
 
 \vspace{-0.6cm} % Agregar 1 cm de espacio entre el párrafo y la figura
 
 En la figura \ref{fig:cali02} se ve el panel inciial del sistema.
 
 \vspace{0.3cm} % Agregar 1 cm de espacio entre el párrafo y la figura
 
 \begin{figure}[!h] % 'H' del paquete 'float' para mantener posición	
 	\caption[Creación de clientes]
 	{\newline Creación de clientes.} % Leyenda en la parte superior
 	\centering
 	\includegraphics[width=0.85\textwidth]{imagenes/cap_3/Img_calibus/CALIBUS111.png} % Inserta una imagen
 	
 	\begin{flushleft}
 		\hspace{1.20cm} \textbf{Nota.} Pantalla de creación de clientes. % Nota al pie para esta figura
 	\end{flushleft}
 	\vspace{-16pt}
 	\label{fig:cali08} % Etiqueta para referencia cruzada
 \end{figure}
 
 En la Figura \ref{fig:cali08} y \ref{fig:cali09} se observa la gestión de los clientes tanto para la creación como el listado de los clientes que estan registrados.
 
 \begin{figure}[!h] % 'H' del paquete 'float' para mantener posición	
 	\caption[Listado de clientes]
 	{\newline Listado de clientes.} % Leyenda en la parte superior
 	\centering
 	\includegraphics[width=0.85\textwidth]{imagenes/cap_3/Img_calibus/CALIBUS112.png} % Inserta una imagen
 	
 	\begin{flushleft}
 		\hspace{1.20cm} \textbf{Nota.} Vista con el listado de clientes registrados. % Nota al pie para esta figura
 	\end{flushleft}
 	\vspace{-16pt}
 	\label{fig:cali09} % Etiqueta para referencia cruzada
 \end{figure}
 
 \vspace{0.6cm} % Agregar 1 cm de espacio entre el párrafo y la figura
  
 \begin{figure}[!h] % 'H' del paquete 'float' para mantener posición	
 	\caption[Creación de un viaje]
 	{\newline Creación de un viaje.} % Leyenda en la parte superior
 	\centering
 	\includegraphics[width=0.83\textwidth]{imagenes/cap_3/Img_calibus/CALIBUS1112.png} % Inserta una imagen
 	
 	\begin{flushleft}
 		\hspace{1.20cm} \textbf{Nota.} Interfaz para la creación de viajes. % Nota al pie para esta figura
 	\end{flushleft}
 	\vspace{-16pt}
 	\label{fig:cali23} % Etiqueta para referencia cruzada
 \end{figure}
 
 En la figura \ref{fig:cali23} se ve la pantalla para la creación de los viajes, posteriormente en la Figura \ref{fig:cali18} se observa la pantalla para la venta de pasajes.
 
 \vspace{0.3cm} % Agregar 1 cm de espacio entre el párrafo y la figura 
 
 \begin{figure}[!h] % 'H' del paquete 'float' para mantener posición	
 	\caption[Venta o reserva de pasajes]
 	{\newline Venta o reserva de pasajes.} % Leyenda en la parte superior
 	\centering
 	\includegraphics[width=0.83\textwidth]{imagenes/cap_3/Img_calibus/CALIBUS113.png} % Inserta una imagen
 	
 	\begin{flushleft}
 		\hspace{1.20cm} \textbf{Nota.} Pantalla para realizar la venta o reserva de pasajes, incluyendo selección de asiento y datos del pasajero. % Nota al pie para esta figura
 	\end{flushleft}
 	\vspace{-16pt}
 	\label{fig:cali18} % Etiqueta para referencia cruzada
 \end{figure}
 
 \vspace{-0.6cm} % Agregar 1 cm de espacio entre el párrafo y la figura
 
 En la figura \ref{fig:cali19} se observa la pantalla de registro de encomiendas, posteriormente en la Figura \ref{fig:cali20} se observa informe de ventas según una fecha específica.
 
 \vspace{0.3cm} % Agregar 1 cm de espacio entre el párrafo y la figura
 
 \begin{figure}[!h] % 'H' del paquete 'float' para mantener posición	
 	\caption[Gestión de encomiendas]
 	{\newline Gestión de encomiendas.} % Leyenda en la parte superior
 	\centering
 	\includegraphics[width=0.83\textwidth]{imagenes/cap_3/Img_calibus/CALIBUS114.png} % Inserta una imagen
 	
 	\begin{flushleft}
 		\hspace{1.20cm} \textbf{Nota.} Módulo para el registro de las encomiendas. % Nota al pie para esta figura
 	\end{flushleft}
 	\vspace{-16pt}
 	\label{fig:cali19} % Etiqueta para referencia cruzada
 \end{figure} 
 
 \begin{figure}[!h] % 'H' del paquete 'float' para mantener posición	
 	\caption[Informe de ventas]
 	{\newline Informe de ventas.} % Leyenda en la parte superior
 	\centering
 	\includegraphics[width=0.83\textwidth]{imagenes/cap_3/Img_calibus/CALIBUS19.png} % Inserta una imagen
 	
 	\begin{flushleft}
 		\hspace{1.20cm} \textbf{Nota.} Informe de las ventas realizadas en el sistema. % Nota al pie para esta figura
 	\end{flushleft}
 	\vspace{-16pt}
 	\label{fig:cali20} % Etiqueta para referencia cruzada
 \end{figure}
 
 \vspace{-0.6cm} % Agregar 1 cm de espacio entre el párrafo y la figura
	
\section{RESULTADOS}
	La implementación del nuevo sistema ha permitido mejorar los tiempos asociados a las tareas del proceso operativo. Este avance se traduce en una mayor eficiencia en las operaciones diarias, ya que se reducen considerablemente los tiempos manuales involucrados en cada actividad.
	
	Además, la reducción de estos tiempos contribuye a minimizar errores humanos, lo que favorece la precisión y confiabilidad de los procesos dentro de la empresa. La automatización facilita que las tareas se realicen de forma más rápida y consistente.
	
	En las tablas 3.12, 3.13, 3.14, 3.15 y 3.16 se presentan comparativas de los tiempos promedio empleados en las principales actividades. Estas tablas reflejan claramente el impacto positivo del nuevo sistema frente a los métodos tradicionales, demostrando su efectividad en la simplificación y mejora continua de las operaciones diarias.
	
	\begingroup
		\onehalfspacing	
		\begin{longtable}{>{\centering\arraybackslash}m{3cm} >{\centering\arraybackslash}m{5cm} >{\centering\arraybackslash}m{4cm}}
			\caption[Tiempos para Registro de Usuarios]{\newline Tiempos para Registro de Usuarios} \label{tab:tabla3_12}\\
			\toprule
			\textbf{Nro} & \textbf{Sistema manual (min:seg)} & \textbf{Sistema web (min:seg)}\\
			\midrule
			\endfirsthead
			\bottomrule
			\endlastfoot
			
			% Aquí se colocan las filas de la tabla, por ejemplo:
			1 & 03:35 & 01:23 \\
			2 & 03:40 & 01:40 \\
			3 & 03:05 & 01:50 \\
			4 & 03:12 & 01:30 \\
			5 & 04:01 & 02:01 \\ \hline
			Promedio	& 03:31 & 01:41 \\ \hline
			Diferencia  &  & 01:50 \\ \hline
			Diferencia (\%) &   & 52.14\% \\
			
		\end{longtable}
		\vspace{-12pt}  % O el valor que necesites para ajustar
		% Nota personalizada fuera de `\caption*{}`
		\textbf{Nota}: Registro de tiempos promedio requeridos para completar el proceso de creación de usuarios en el sistema.
	\endgroup
	
	\begingroup
		\onehalfspacing	
		\begin{longtable}{>{\centering\arraybackslash}m{3cm} >{\centering\arraybackslash}m{5cm} >{\centering\arraybackslash}m{4cm}}
			\caption[Tiempos para Venta de Pasajes]{\newline Tiempos para Venta de Pasajes} \label{tab:tabla3_13}\\
			\toprule
			\textbf{Nro} & \textbf{Sistema manual (min:seg)} & \textbf{Sistema web (min:seg)}\\
			\midrule
			\endfirsthead
			\bottomrule
			\endlastfoot	
			% Aquí se colocan las filas de la tabla, por ejemplo:
			1 & 04:10 & 01:30 \\	
			2 & 04:39 & 01:20 \\
			3 & 05:16 & 01:26 \\
			4 & 04:28 & 01:10 \\
			5 & 04:36 & 01:45 \\ \hline
			Promedio	& 04:38 & 01:26 \\ \hline
			Diferencia  &  & 03:12 \\ \hline
			Diferencia (\%) &   & 69\% \\
			
		\end{longtable}
		\vspace{-12pt}  % O el valor que necesites para ajustar
		% Nota personalizada fuera de `\caption*{}`
		\textbf{Nota}: Medición de tiempos promedio requeridos para completar el proceso de creación de usuarios en el sistema.
	\endgroup
	
	\begingroup
		\onehalfspacing	
		\begin{longtable}{>{\centering\arraybackslash}m{3cm} >{\centering\arraybackslash}m{5cm} >{\centering\arraybackslash}m{4cm}}
			\caption[Tiempos para Reserva de Pasajes]{\newline Tiempos para Reserva de Pasajes} \label{tab:tabla3_14}\\
			\toprule
			\textbf{Nro} & \textbf{Sistema manual (min:seg)} & \textbf{Sistema web (min:seg)}\\
			\midrule
			\endfirsthead
			\bottomrule
			\endlastfoot			
			% Aquí se colocan las filas de la tabla, por ejemplo:
			1 & 02:15 & 00:58 \\
			2 & 02:01 & 00:56 \\
			3 & 02:21 & 00:49 \\
			4 & 02:32 & 00:52 \\
			5 & 02:03 & 00:56 \\ \hline
			Promedio	& 02:14 & 00:54 \\ \hline
			Diferencia  &  & 01:20 \\ \hline
			Diferencia (\%) &   & 59.67\% \\
			
		\end{longtable}
		\vspace{-12pt}  % O el valor que necesites para ajustar
		% Nota personalizada fuera de `\caption*{}`
		\textbf{Nota}: Resultados de los tiempos de respuesta para completar la reserva de pasajes en distintas pruebas.
	\endgroup
	
	\begingroup
		\onehalfspacing	
	\begin{longtable}{>{\centering\arraybackslash}m{3cm} >{\centering\arraybackslash}m{5cm} >{\centering\arraybackslash}m{4cm}}
		\caption[Tiempos para Registro de Encomiendas]{\newline Tiempos para Registro de Encomiendas} \label{tab:tabla3_15}\\
		\toprule
		\textbf{Nro} & \textbf{Sistema manual (min:seg)} & \textbf{Sistema web (min:seg)}\\
		\midrule
		\endfirsthead
		\bottomrule
		\endlastfoot		
		% Aquí se colocan las filas de la tabla, por ejemplo:
		1 & 04:12 & 02:23 \\
		2 & 05:01 & 02:11 \\
		3 & 04:26 & 02:02 \\
		4 & 04:45 & 02:03 \\
		5 & 05:15 & 02:43 \\ \hline
		Promedio	& 04:44 & 02:16 \\ \hline
		Diferencia  &  & 02:27 \\ \hline
		Diferencia (\%) &   & 51.94\% \\
		
	\end{longtable}
	\vspace{-12pt}  % O el valor que necesites para ajustar
	% Nota personalizada fuera de `\caption*{}`
	\textbf{Nota}: Tiempos necesarios para registrar una encomienda desde su ingreso hasta la confirmación.
	\endgroup
	
	\begingroup
		\onehalfspacing	
	\begin{longtable}{>{\centering\arraybackslash}m{3cm} >{\centering\arraybackslash}m{5cm} >{\centering\arraybackslash}m{4cm}}
		\caption[Tiempos para Generar Reportes]{\newline Tiempos para Generar Reportes} \label{tab:tabla3_16}\\
		\toprule
		\textbf{Nro} & \textbf{Sistema manual (min:seg)} & \textbf{Sistema web (min:seg)}\\
		\midrule
		\endfirsthead	
		\bottomrule
		\endlastfoot
		% Aquí se colocan las filas de la tabla, por ejemplo:
		1 & 20:14 & 02:03 \\
		2 & 21:10 & 02:22 \\
		3 & 19:56 & 02:01 \\
		4 & 19:22 & 01:59 \\
		5 & 23:45 & 01:39 \\ \hline
		Promedio	& 20:53 & 02:01 \\ \hline
		Diferencia  &  & 18:53 \\ \hline
		Diferencia (\%) &   & 90.36\% \\
		
	\end{longtable}
	\vspace{-12pt}  % O el valor que necesites para ajustar
	% Nota personalizada fuera de `\caption*{}`
	\textbf{Nota}: Evaluación de los tiempos requeridos para le generación de reportes por parte del sistema.
	\endgroup

	\begingroup
	\onehalfspacing	
	\begin{longtable}{>{\centering\arraybackslash}m{3.5cm} >{\centering\arraybackslash}m{3.5cm} >{\centering\arraybackslash}m{3cm} >{\centering\arraybackslash}m{2.5cm} >{\centering\arraybackslash}m{2.5cm}}
		\caption[Resumen tiempos de actividades]{\newline Resumen tiempos de actividades} \label{tab:tabla3_17}\\
		\toprule
		\textbf{Actividad} & \textbf{Sistema manual (min:seg)} & \textbf{Sistema web (min:seg)} & \textbf{Diferencia} & \textbf{Promedio} \% \\
		\midrule
		\endfirsthead
		\bottomrule
		\endlastfoot	
		% Aquí se colocan las filas de la tabla, por ejemplo:
		Registro de usuarios 	& 03:31 & 01:41 & 01:50 & 52.14 \\
		Ventas de pasajes 		& 04:38 & 01:26 & 03:12 & 68.97 \\
		Reserva de pasajes 		& 02:14 & 00:54 & 01:20 & 59.67 \\
		Registro de encomiendas & 04:44 & 02:16 & 02:27 & 51.94 \\
		Generar reportes 		& 20:53 & 02:01 & 18:53 & 90.36 \\ \hline
		Promedio				&       &       &       & 64.62 \\
		
	\end{longtable}
	\vspace{-12pt}  % O el valor que necesites para ajustar
	% Nota personalizada fuera de `\caption*{}`
	\textbf{Nota}: Comparativo general de los tiempos empleados en cada una de las actividades principales del sistema.
	\endgroup

	\vspace{0.3cm} % Agregar 1 cm de espacio entre el párrafo y la figura
	
	\begin{figure}[h] % 'H' del paquete 'float' para mantener posición	
			\caption[Comparación de mejora de tiempos]
			{\newline Comparación de mejora de tiempos.} % Leyenda en la parte superior
			\centering
			\includegraphics[width=0.8\textwidth]{imagenes/cap_3/Resultados.png} % Inserta una imagen
			
		\begin{flushleft}
			\begin{doublespace}
				\hspace{1.20cm} \textbf{Nota.} La figura muestra la diferencia de tiempos promedio antes y despues de la implementación del sistema. % Nota al pie para esta figura
			\end{doublespace}
		\end{flushleft}
		\vspace{-16pt} % hace que se acerque mas el texto
		\label{fig:figura_resultados} % Etiqueta para referencia cruzada
	\end{figure}
	
	Dados los resultados de la tabla \ref{tab:tabla3_17} y la figura \ref{fig:figura_resultados} se evidencia que la implementación del nuevo sistema permite mejorar los procesos operativos, logrando una reducción promedio del 64.62 por ciento en los tiempos de ejecución de actividades, este ahorro representa una transformación significativa en la mejora dela empresa, ya que los procesos manuales, que antes eran tediosos y propensos a errores, han sido sustituidos por un sistema automatizado que prioriza la rapidez. Este avance no solo mejora la experiencia del cliente al reducir los tiempos de espera, sino que también libera recursos para que el personal pueda enfocarse en actividades de mayor valor, fortaleciendo la posición competitiva de la empresa en el mercado.
	% Calidad y seguridad
	\chapter{CALIDAD Y SEGURIDAD} 
\section{CALIDAD DE SOFTWARE}
\section{NORMAS DE CALIDAD}
\subsection{Usabilidad}

La usabilidad se evaluará a través de la escala de Likert, la cual permitirá medir la cantidad de clics necesarios para realizar distintas tareas en el software. De esta manera, se podrá valorar el nivel de satisfacción de los usuarios respecto al número de clics requeridos para completar dichas tareas. Además, esta escala proporcionará una visión clara sobre la eficiencia del diseño del software, ayudando a identificar áreas que podrían necesitar ajustes para mejorar la experiencia del usuario. La escala se define de la siguiente manera:

\begin{itemize}[label=$-$, left=0cm, labelsep = 0.9cm, topsep = 0pt, parsep = 0pt]
	\item \textbf{Muy insatisfecho:} Cinco o más clics, ya que las tareas requieren demasiados clics y son consideradas tediosas.		
	\item \textbf{Insatisfecho:} Cuatro clics, lo que indica que la tarea requiere más clics de los deseados y resulta algo incómoda.
	\item \textbf{Neutral:} Tres clics, una cantidad aceptable, aunque con margen de mejora.
	\item \textbf{Satisfecho:} Dos clics, considerado un número razonable que permite realizar la tarea de manera eficiente.
	\item \textbf{Muy satisfecho:} Un clic, ideal por ser altamente eficiente y requerir el mínimo esfuerzo.
\end{itemize}

\begin{longtable}{>{\centering\arraybackslash}m{5cm} >{\centering\arraybackslash}m{3cm} >{\centering\arraybackslash}m{3cm}}
	\caption[Número de Clics - Escala Likert]{\newline Número de Clics - Escala Likert} \label{tab:tabla_clics}\\
	\toprule
	\textbf{Tarea} & \textbf{Clics} & \textbf{Puntos}\\
	\midrule
	\endfirsthead
	
	\toprule
	\textbf{Tarea} & \textbf{Clics} & \textbf{Puntos}\\
	\midrule
	\endhead
	
	%\midrule
	%\multicolumn{3}{r}{\textit{Continúa en la siguiente página}} \\
	%\midrule
	%\endfoot
	
	\bottomrule
	\endlastfoot
	
	% Aquí se colocan las filas de la tabla, por ejemplo:
	Ingresar usuario      & 2 & 2 \\
	Registrar usuario     & 3 & 3 \\
	Registrar cliente     & 4 & 4 \\
	Venta de pasaje       & 3 & 3 \\
	Cancelar pasaje       & 2 & 2 \\
	Registrar encomienda  & 4 & 4 \\
	Generar reporte		  & 2 & 2 \\
	Cerrar sesión		  & 1 & 1 \\ \hline
	Total				  & 21 &  \\ \hline
	Promedio 			  &   & 2.62 \\
	
\end{longtable}
\vspace{-12pt}  % O el valor que necesites para ajustar
% Nota personalizada fuera de `\caption*{}`
% \textbf{Nota}: Esta es la nota de la tabla, explicando datos relevantes.

Se tiene un total de 21 clics para las tareas realizadas, con esto número obtenemos el promedio de clics de 2.62 y con este resultado concluimos que la usabilidad del sistema es "Satisfactorio".

\subsection{Portabilidad}

El proyecto es compatible con navegadores modernos asegurando una experiencia uniforme mediante el uso de estándares web, esto permite que el sistema funcione de manera consistente en distintos entornos, facilitando su acceso desde cualquier dispositivo. Además, se han implementado medidas para optimizar el rendimiento y minimizar inconsistencias entre navegadores, garantizando una experiencia fluida y eficiente para todos los usuarios, como podemos ver en las figuras \ref{fig:figura_brave}, \ref{fig:figura_edge} y \ref{fig:figura_celular}.

\vspace{3cm} % Agregar 1 cm de espacio entre el párrafo y la figura

\begin{figure}[!h] % 'H' del paquete 'float' para mantener posición	
	\caption[Navegador Brave]
	{\newline Navegador Brave.} % Leyenda en la parte superior
	\centering
	\includegraphics[width=0.95\textwidth]{imagenes/cap_3/brave.png} % Inserta una imagen
	
	%\begin{flushleft}
	%	\hspace{1.20cm} \textit{Nota.} al pie asociada con esta figura, explicando detalles adicionales. % Nota al pie para esta figura
	%\end{flushleft}
	% \vspace{-16pt}
	\label{fig:figura_brave} % Etiqueta para referencia cruzada
\end{figure}

\vspace{1cm} % Agregar 1 cm de espacio entre el párrafo y la figura

\begin{figure}[!h] % 'H' del paquete 'float' para mantener posición	
	\caption[Navegador Microsoft Edge]
	{\newline Navegador Microsoft Edge.} % Leyenda en la parte superior
	\centering
	\includegraphics[width=0.95\textwidth]{imagenes/cap_3/edge.png} % Inserta una imagen
	
	%\begin{flushleft}
	%	\hspace{1.20cm} \textit{Nota.} al pie asociada con esta figura, explicando detalles adicionales. % Nota al pie para esta figura
	%\end{flushleft}
	\vspace{16pt}
	\label{fig:figura_edge} % Etiqueta para referencia cruzada
\end{figure}

\vspace{6cm} % Agregar 1 cm de espacio entre el párrafo y la figura

\begin{figure}[!h] % 'H' del paquete 'float' para mantener posición	
	\caption[Navegador Chrome - Móvil]
	{\newline Navegador Chrome - Móvil.} % Leyenda en la parte superior
	\centering
	\includegraphics[width=0.4\textwidth]{imagenes/cap_3/celular.png} % Inserta una imagen
	
	%\begin{flushleft}
	%	\hspace{1.20cm} \textit{Nota.} al pie asociada con esta figura, explicando detalles adicionales. % Nota al pie para esta figura
	%\end{flushleft}
	\vspace{-16pt}
	\label{fig:figura_celular} % Etiqueta para referencia cruzada
\end{figure}

\vspace{0.3cm} % Agregar 1 cm de espacio entre el párrafo y la figura


\section{SEGURIDAD}

Implementar un sistema en línea accesible para los clientes, que permita realizar la compra de pasajes y la gestión de reservas desde cualquier dispositivo conectado a Internet. Esto no solo facilitará el acceso a los servicios de la empresa, sino que también incrementará su competitividad en el mercado al adaptarse a las expectativas modernas de los usuarios.

Realizar backups semestrales de la base de datos, garantizando que toda la información del sistema esté protegida frente a posibles pérdidas o fallos. Estos respaldos deben almacenarse en ubicaciones seguras, preferentemente en servidores externos o en la nube, para asegurar la recuperación efectiva de datos en caso de emergencias.

Desarrollar un sistema de chat en línea que brinde soporte en tiempo real a los usuarios, permitiéndoles resolver dudas, reportar problemas o recibir asistencia durante el proceso de compra y reserva. Este canal de comunicación directa mejorará la experiencia del cliente y reforzará la percepción de calidad del servicio ofrecido por la empresa.

Evaluar la posibilidad de automatizar las notificaciones a los clientes a través de correos electrónicos o mensajes de texto, informándoles sobre el estado de sus reservas o encomiendas, así como cualquier actualización en los servicios de la empresa. Esto contribuirá a mantener una comunicación efectiva con los usuarios.

\begin{figure}[!h] % 'H' del paquete 'float' para mantener posición	
	\caption[Interfaz gráfica - Cambio de contraseña]
	{\newline Interfaz gráfica - Cambio de contraseña.} % Leyenda en la parte superior
	\centering
	\includegraphics[width=0.85\textwidth]{imagenes/cap_3/Img_calibus/CALIBUS09.png} % Inserta una imagen
	
	\begin{flushleft}
		%\hspace{1.20cm} \textbf{Nota.} Organigrama obtenido en entrevista con el administrador. % Nota al pie para esta figura
	\end{flushleft}
	\vspace{-16pt}
	\label{fig:cali40} % Etiqueta para referencia cruzada
\end{figure}

\vspace{-0.6cm} % Agregar 1 cm de espacio entre el párrafo y la figura


 \begin{figure}[!h] % 'H' del paquete 'float' para mantener posición	
	\caption[Interfaz gráfica - Lista de accesos]
	{\newline Interfaz gráfica - Lista de accesos.} % Leyenda en la parte superior
	\centering
	\includegraphics[width=0.83\textwidth]{imagenes/cap_3/Img_calibus/CALIBUS18.png} % Inserta una imagen
	
	\begin{flushleft}
		%\hspace{1.20cm} \textbf{Nota.} Organigrama obtenido en entrevista con el administrador. % Nota al pie para esta figura
	\end{flushleft}
	\vspace{-16pt}
	\label{fig:cali41} % Etiqueta para referencia cruzada
\end{figure}

\vspace{-0.6cm} % Agregar 1 cm de espacio entre el párrafo y la figura

	% jConclusiones y recomendaciones
	\chapter{CONCLUSIONES Y RECOMENDACIONES} 
\section{CONCLUSIONES}
	El análisis de los procesos de la venta de pasajes y envío de encomiendas en la empresa de transportes Cali Internacional permitió identificar áreas para la mejora, los procesos manuales en la venta de pasajes generaban demoras que afectaban tanto a la satisfacción de los clientes como a la operatividad interna. Con el diseño del nuevo sistema, se lograron agilizar estos procesos, automatizando las tareas más repetitivas y reduciendo los márgenes de error. 
	
	La interfaz esta centrada en la experiencia del usuario, se priorizó la simplicidad, la interfaz facilita la interacción con el sistema, permite a los usuarios navegar por las opciones de venta, reserva de pasajes y envío de encomiendas sin dificultades.
	
	La base de datos normalizada hasta la tercera forma normal proporcionó una estructura sólida para el almacenamiento y gestión de la información. Además, las consultas han resultado en una mejora en los tiempos de respuesta del sistema, beneficiando directamente la operatividad diaria de la empresa.
	
	El back-end del sistema, enfocado en la integración con la base de datos, realizó una buena gestión de las operaciones relacionadas con la venta de pasajes y el envío de encomiendas, gracias a una estructura bien definida y una ejecución correcta de las operaciones, los usuarios pueden realizar transacciones en tiempo real con la certeza de que la información está actualizada.
	
	La implementación de herramientas de generación de reportes hace que la administración de la empresa tome decisiones basadas en datos actualizados, la capacidad de obtener informes detallados sobre ventas y el estado de las encomiendas brinda a los administradores la información necesaria para ajustar la asignación de recursos y la planificación estratégica. En conclusión, el proyecto logró cumplir con los objetivos establecidos, proporcionando a la empresa una solución tecnológica integral que mejora la operatividad, la experiencia del usuario y la toma de decisiones.

\section{RECOMENDACIONES}

	Implementar un sistema en línea accesible para los clientes, que permita realizar la compra de pasajes y la gestión de reservas desde cualquier dispositivo conectado a Internet. Esto no solo facilitará el acceso a los servicios de la empresa, sino que también incrementará su competitividad en el mercado al adaptarse a las expectativas modernas de los usuarios.
	
	Realizar backups semestrales de la base de datos, garantizando que toda la información del sistema esté protegida frente a posibles pérdidas o fallos. Estos respaldos deben almacenarse en ubicaciones seguras, preferentemente en servidores externos o en la nube, para asegurar la recuperación efectiva de datos en caso de emergencias.
	
	Desarrollar un sistema de chat en línea que brinde soporte en tiempo real a los usuarios, permitiéndoles resolver dudas, reportar problemas o recibir asistencia durante el proceso de compra y reserva. Este canal de comunicación directa mejorará la experiencia del cliente y reforzará la percepción de calidad del servicio ofrecido por la empresa.
	
	Evaluar la posibilidad de automatizar las notificaciones a los clientes a través de correos electrónicos o mensajes de texto, informándoles sobre el estado de sus reservas o encomiendas, así como cualquier actualización en los servicios de la empresa. Esto contribuirá a mantener una comunicación efectiva con los usuarios.
	
	
	\backmatter
	% Bibliografía
	\chapter*{BIBLIOGRAFÍA}
\nocite{cuevas2023}
\nocite{cuevas2024}
\printbibliography[heading = none]

\end{document}