\chapter{MARCO APLICATIVO}
\subsection{Requerimientos funcionales}
Los requerimientos funcionales describen lo que el sistema debe hacer, es decir, las funciones o características específicas que se deben implementar. En este proyecto, los requerimientos funcionales incluirían aspectos como:
\begin{itemize}[label=$\bullet$, left=0cm, labelsep = 1.05cm, topsep = 0pt, parsep = 0pt]
	\item Gestión de rutas y horarios
	\item Venta y reserva de boletos
	\item Asignación de asientos
	\item Registro y seguimiento de encomiendas
	\item Gestión de flota de buses
	\item Generación de reportes de ventas
	\item Manejo de perfiles de usuarios (administradores, empleados)
\end{itemize}
\subsection{Requerimientos no funcionales}
Por otro lado, los requerimientos no funcionales especifican cómo debe funcionar el sistema en términos de rendimiento, usabilidad, seguridad y confiabilidad. Para el sistema, algunos de los requerimientos no funcionales incluyen:
\begin{itemize}[label=$\bullet$, left=0cm, labelsep = 1.05cm, topsep = 0pt, parsep = 0pt]
	\item El sistema debe ser accesible desde múltiples dispositivos (computadoras, tablets y móviles).
	\item Garantizar la protección de los datos de los usuarios mediante protocolos de seguridad adecuados.
	\item Escalabilidad, capacidad de crecer para manejar aumento en usuarios y transacciones
	\item Mantenibilidad, arquitectura modular para facilitar actualizaciones
\end{itemize}