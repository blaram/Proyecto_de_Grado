% Contenido del resumen
\chapter*{} % Deja el capítulo sin título aquí
\addcontentsline{toc}{chapter}{RESUMEN}

\begin{center}
	{\bfseries\fontsize{13pt}{16pt}\selectfont RESUMEN}
\end{center}

\vspace{1em}

El proyecto titulado “Software de logística y gestión de buses para transporte de pasajeros y envío de encomiendas” tiene como objetivo implementar un sistema computarizado que optimice los servicios de venta y reserva de pasajes, así como la gestión de la recepción y entrega de encomiendas en la empresa. Durante el análisis de procesos, se identificó que los registros de pasajes y encomiendas se realizaban de forma manual, lo que resultaba en un uso ineficiente del tiempo y los recursos materiales.

El desarrollo del sistema se fundamentó en un análisis detallado de las actividades de la empresa Cali Internacional, empleando el modelo en cascada como metodología principal. Los requisitos funcionales y no funcionales fueron obtenidos a través de observaciones directas y entrevistas con el personal. Para la implementación técnica, se seleccionó el framework Django de Python, aprovechando sus capacidades de programación orientada a objetos, junto con PostgreSQL como gestor de base de datos, debido a su excelente integración con Django.

La finalización del proyecto permitió una notable optimización de las operaciones, reflejada en una reducción del 64.62 por ciento en los tiempos de procesamiento. Las actividades relacionadas con el registro y la venta de boletos se agilizaron considerablemente, mejorando la atención al cliente y haciendo el proceso más eficiente. Este avance tuvo un impacto positivo en la productividad diaria y la calidad del servicio brindado.

\vspace{1em}
\noindent\textbf{Palabras clave:} pasajes, encomiendas, Python, Django.
