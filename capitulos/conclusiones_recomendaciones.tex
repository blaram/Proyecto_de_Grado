\chapter{CONCLUSIONES Y RECOMENDACIONES} 
\section{CONCLUSIONES}
	El análisis de los procesos de la venta de pasajes y envío de encomiendas en la empresa de transportes Cali Internacional permitió identificar áreas para la mejora, los procesos manuales en la venta de pasajes generaban demoras que afectaban tanto a la satisfacción de los clientes como a la operatividad interna. Con el diseño del nuevo sistema, se lograron agilizar estos procesos, automatizando las tareas más repetitivas y reduciendo los márgenes de error. 
	
	La interfaz esta centrada en la experiencia del usuario, se priorizó la simplicidad, la interfaz facilita la interacción con el sistema, permite a los usuarios navegar por las opciones de venta, reserva de pasajes y envío de encomiendas sin dificultades.
	
	La base de datos normalizada hasta la tercera forma normal proporcionó una estructura sólida para el almacenamiento y gestión de la información. Además, las consultas han resultado en una mejora en los tiempos de respuesta del sistema, beneficiando directamente la operatividad diaria de la empresa.
	
	El back-end del sistema, enfocado en la integración con la base de datos, realizó una buena gestión de las operaciones relacionadas con la venta de pasajes y el envío de encomiendas, gracias a una estructura bien definida y una ejecución correcta de las operaciones, los usuarios pueden realizar transacciones en tiempo real con la certeza de que la información está actualizada.
	
	La implementación de herramientas de generación de reportes hace que la administración de la empresa tome decisiones basadas en datos actualizados, la capacidad de obtener informes detallados sobre ventas y el estado de las encomiendas brinda a los administradores la información necesaria para ajustar la asignación de recursos y la planificación estratégica. En conclusión, el proyecto logró cumplir con los objetivos establecidos, proporcionando a la empresa una solución tecnológica integral que mejora la operatividad, la experiencia del usuario y la toma de decisiones.

\section{RECOMENDACIONES}

	Implementar un sistema en línea accesible para los clientes, que permita realizar la compra de pasajes y la gestión de reservas desde cualquier dispositivo conectado a Internet. Esto no solo facilitará el acceso a los servicios de la empresa, sino que también incrementará su competitividad en el mercado al adaptarse a las expectativas modernas de los usuarios.
	
	Realizar backups semestrales de la base de datos, garantizando que toda la información del sistema esté protegida frente a posibles pérdidas o fallos. Estos respaldos deben almacenarse en ubicaciones seguras, preferentemente en servidores externos o en la nube, para asegurar la recuperación efectiva de datos en caso de emergencias.
	
	Desarrollar un sistema de chat en línea que brinde soporte en tiempo real a los usuarios, permitiéndoles resolver dudas, reportar problemas o recibir asistencia durante el proceso de compra y reserva. Este canal de comunicación directa mejorará la experiencia del cliente y reforzará la percepción de calidad del servicio ofrecido por la empresa.
	
	Evaluar la posibilidad de automatizar las notificaciones a los clientes a través de correos electrónicos o mensajes de texto, informándoles sobre el estado de sus reservas o encomiendas, así como cualquier actualización en los servicios de la empresa. Esto contribuirá a mantener una comunicación efectiva con los usuarios.
	