% Contenido del resumen
\chapter*{}
\thispagestyle{plain}  % <- Muestra el número de página en esta hoja
\addcontentsline{toc}{chapter}{ABSTRACT}
\begin{center}
	{\bfseries\fontsize{13pt}{16pt}\selectfont ABSTRACT}
\end{center}

\vspace{1em}

The project entitled “Logistics and bus management software for passenger transport and parcel delivery” aims to implement a computerized system that optimizes ticket sales and reservation services, as well as the reception and delivery of packages. During the process analysis, it was identified that ticket and package records were handled manually, which resulted in inefficient use of time and material resources.

The system development was based on a detailed analysis of the activities of the company Cali Internacional, using the waterfall model as the main methodology. Functional and non-functional requirements were obtained through direct observations and interviews with the staff. For the technical implementation, the Django framework from Python was selected, leveraging its object-oriented programming capabilities, along with PostgreSQL as the database manager, due to its excellent integration with Django.

The completion of the project enabled a significant optimization of operations, reflected in a 64.62 percent reduction in processing times. Activities related to ticket registration and sales were significantly streamlined, improving customer service and making the process more efficient. This advance had a positive impact on daily productivity and service quality.

\vspace{1em}
\noindent\textbf{Keywords:} tickets, parcels, Python, Django.
